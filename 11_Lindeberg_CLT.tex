\documentclass[12pt]{article}


\topmargin 0pt
\advance \topmargin by -\headheight
\advance \topmargin by -\headsep
\textheight 8.9in
\oddsidemargin 0pt
\evensidemargin \oddsidemargin
\marginparwidth 0.5in
\textwidth 6.5in

\parindent 0in
\parskip 1.5ex
%\renewcommand{\baselinestretch}{1.25}

\usepackage{xr-hyper}
\def\renewtheorem{}
\usepackage{lmodern}

% Note: this has been tested using MiKTeX 2.9. If you are getting errors, update your packages.
% LTeX: enabled=false

%%% Packages %%%
%\usepackage{setspace} % Double spaces document. Footnotes,
% figures, and tables will still be single spaced, however.
%\doublespacing
%\singlespacing
%\onehalfspacing
% \setstretch{1.5} % set double spacing to 1.5 or anything else.

\usepackage[T1]{fontenc}
\usepackage[utf8]{inputenc}
\usepackage{amsmath,amssymb,amsfonts,mathrsfs,bm}% Typical maths resource packages
\usepackage{mathtools}
\usepackage{amsthm}
\usepackage{nicefrac}
\usepackage[shortlabels]{enumitem}
\usepackage{graphicx}
\usepackage{epstopdf}
\DeclareGraphicsExtensions{.eps,.png,.jpg,.pdf}

\usepackage{url}
\usepackage{colortbl}
\usepackage{booktabs}
\usepackage{multirow}
\usepackage[table,dvipsnames]{xcolor}
\usepackage[normalem]{ulem}
\usepackage{xparse}
\usepackage{calc}
\usepackage{etoolbox}

\makeatletter
\@ifpackageloaded{natbib}{
	\relax
}{
	\usepackage{cite}
}
\makeatother

%\usepackage{pstricks}
%\usepackage{psfrag}
%\usepackage{syntonly}
%\syntaxonly
%\usepackage[style=base]{caption}
%\captionsetup{
%format = plain,
%font = footnotesize,
%labelfont = sc
%}


\usepackage{array}
\newcolumntype{L}[1]{>{\raggedright\let\newline\\\arraybackslash\hspace{0pt}}m{#1}}
\newcolumntype{C}[1]{>{\centering\let\newline\\\arraybackslash\hspace{0pt}}m{#1}}
\newcolumntype{R}[1]{>{\raggedleft\let\newline\\\arraybackslash\hspace{0pt}}m{#1}}

\makeatletter
\let\MYcaption\@makecaption
\makeatother
\usepackage[font=footnotesize]{subcaption}
\makeatletter
\let\@makecaption\MYcaption
\makeatother

\usepackage{glossaries}
\makeatletter
% copy old \gls and \glspl
\let\oldgls\gls
\let\oldglspl\glspl

% define a non space skipping version of \@ifnextchar
\newcommand\fussy@ifnextchar[3]{%
	\let\reserved@d=#1%
	\def\reserved@a{#2}%
	\def\reserved@b{#3}%
	\futurelet\@let@token\fussy@ifnch}
\def\fussy@ifnch{%
	\ifx\@let@token\reserved@d
		\let\reserved@c\reserved@a
	\else
		\let\reserved@c\reserved@b
	\fi
	\reserved@c}

\renewcommand{\gls}[1]{%
\oldgls{#1}\fussy@ifnextchar.{\@checkperiod}{\@}}
\renewcommand{\glspl}[1]{%
\oldglspl{#1}\fussy@ifnextchar.{\@checkperiod}{\@}}

\newcommand{\@checkperiod}[1]{%
	\ifnum\sfcode`\.=\spacefactor\else#1\fi
}
\makeatother

\newacronym{wrt}{w.r.t.}{with respect to}
\newacronym{RHS}{R.H.S.}{right-hand side}
\newacronym{LHS}{L.H.S.}{left-hand side}
\newacronym{iid}{i.i.d.}{independent and identically distributed}
%\newacronym{MIMO}{MIMO}{mulitple-input multiple-output}
%\newacronym{AOA}{AOA}{angle-of-arrival}
%\newacronym{AOD}{AOD}{angle-of-departure}
%\newacronym{LOS}{LOS}{line-of-sight}
%\newacronym{NLOS}{NLOS}{non-line-of-sight}
%\newacronym{TOA}{TOA}{time-of-arrival}
%\newacronym{TDOA}{TDOA}{time-difference-of-arrival}
%\newacronym{RSS}{RSS}{received signal strength}
%\newacronym{GNSS}{GNSS}{Global Navigation Satellite System}
%\newacronym{GSP}{GSP}{graph signal processing}
%\newacronym{ML}{ML}{machine learning}


%put the float package before hyperref and algorithm package after hyperref for hyperref to work correctly with algorithm
\usepackage{float}

\ifx\notloadhyperref\undefined
	\ifx\loadbibentry\undefined
		\usepackage[hidelinks,hypertexnames=false]{hyperref}
	\else
		\usepackage{bibentry}
		\makeatletter\let\saved@bibitem\@bibitem\makeatother
		\usepackage[hidelinks,hypertexnames=false]{hyperref}
		\makeatletter\let\@bibitem\saved@bibitem\makeatother
	\fi
\else
	\ifx\loadbibentry\undefined
		\relax
	\else
		\usepackage{bibentry}
	\fi
\fi

\usepackage[capitalize]{cleveref}
\crefname{equation}{}{}
\Crefname{equation}{}{}
\crefname{claim}{claim}{claims}
\crefname{step}{step}{steps}
\crefname{line}{line}{lines}
\crefname{condition}{condition}{conditions}
\crefname{dmath}{}{}
\crefname{dseries}{}{}
\crefname{dgroup}{}{}

\crefname{Problem}{Problem}{Problems}
\crefformat{Problem}{Problem~(#2#1#3)}
\crefrangeformat{Problem}{Problems~(#3#1#4) to~(#5#2#6)}

\crefname{Theorem}{Theorem}{Theorems}
\crefname{Corollary}{Corollary}{Corollaries}
\crefname{Proposition}{Proposition}{Propositions}
\crefname{Lemma}{Lemma}{Lemmas}
\crefname{Definition}{Definition}{Definitions}
\crefname{Example}{Example}{Examples}
\crefname{Assumption}{Assumption}{Assumptions}
\crefname{Remark}{Remark}{Remarks}
\crefname{Rem}{Remark}{Remarks}
\crefname{remarks}{Remarks}{Remarks}
\crefname{Appendix}{Appendix}{Appendices}
\crefname{Supplement}{Supplement}{Supplements}
\crefname{Exercise}{Exercise}{Exercises}
\crefname{Theorem_A}{Theorem}{Theorems}
\crefname{Corollary_A}{Corollary}{Corollaries}
\crefname{Proposition_A}{Proposition}{Propositions}
\crefname{Lemma_A}{Lemma}{Lemmas}
\crefname{Definition_A}{Definition}{Definitions}

\usepackage{crossreftools}
\ifx\notloadhyperref\undefined
	\pdfstringdefDisableCommands{%
		\let\Cref\crtCref
		\let\cref\crtcref
	}
\else
	\relax
\fi

\usepackage{algorithm,algorithmic}
\renewcommand{\algorithmicrequire}{\textbf{Input:}}
\renewcommand{\algorithmicensure}{\textbf{Output:}}

%may cause conflict with some packages like tikz, include manually if desired
%load after hyperref
\ifx\loadbreqn\undefined
	\relax
\else
	\usepackage{breqn}
\fi



%%%%%%%%%%%%%%%%%%%%%%%%%%%%%%%%%%%%%%%%%%%%%%%%


\interdisplaylinepenalty=2500   % To restore IEEEtran ability to automatically break
% within multiline equations, when using amsmath.

%%%%%%%%%%%%%%%%%%%%%%%%%%%%%%%%%%%%%%%%

%Theorem declarations

\ifx\renewtheorem\undefined
	% for use in main body
	\ifx\useTheoremCounter\undefined
		\newtheorem{Theorem}{Theorem}
		\newtheorem{Corollary}{Corollary}
		\newtheorem{Proposition}{Proposition}
		\newtheorem{Lemma}{Lemma}
	\else
		\newtheorem{Theorem}{Theorem}
		\newtheorem{Corollary}[Theorem]{Corollary}
		\newtheorem{Proposition}[Theorem]{Proposition}
	\fi

	\newtheorem{Definition}{Definition}
	\newtheorem{Example}{Example}
	\newtheorem{Remark}{Remark}
	\newtheorem{Assumption}{Assumption}
	\newtheorem{Exercise}{Exercise}

	% for use in the appendix
	\newtheorem{Theorem_A}{Theorem}[section]
	\newtheorem{Corollary_A}{Corollary}[section]
	\newtheorem{Proposition_A}{Proposition}[section]
	\newtheorem{Lemma_A}{Lemma}[section]
	\newtheorem{Definition_A}{Definition}[section]
	\newtheorem{Example_A}{Example}[section]
	\newtheorem{Remark_A}{Remark}[section]
	\newtheorem{Assumption_A}{Assumption}[section]
	\newtheorem{Exercise_A}{Exercise}[section]
\fi

% Remarks
\theoremstyle{remark}
\newtheorem{Rem}{Remark}
\theoremstyle{plain}

\newenvironment{remarks}{
	\begin{list}{\textit{Remark} \arabic{Rem}:~}{
			\setcounter{enumi}{\value{Rem}}
			\usecounter{Rem}
			\setcounter{Rem}{\value{enumi}}
			\setlength\labelwidth{0in}
			\setlength\labelsep{0in}
			\setlength\leftmargin{0in}
			\setlength\listparindent{0in}
			\setlength\itemindent{15pt}
		}
		}{
	\end{list}
}


% Special Headings
%\newtheorem*{Prop1}{Proposition 1} %needs amsthm

%\newtheoremstyle{nonum}{}{}{\itshape}{}{\bfseries}{.}{ }{#1 (\mdseries #3)}
%\theoremstyle{nonum}
%\newtheorem{Example**}{Example 1}

\newcommand{\EndExample}{{$\square$}}
%\renewcommand{\QED}{\QEDopen} % changes end of proof box to open box.

\newcommand{\qednew}{\nobreak \ifvmode \relax \else
		\ifdim\lastskip<1.5em \hskip-\lastskip
			\hskip1.5em plus0em minus0.5em \fi \nobreak
		\vrule height0.75em width0.5em depth0.25em\fi}


% achieves the functionality of \tag for subequations environment
\makeatletter
\newenvironment{varsubequations}[1]
{%
	\addtocounter{equation}{-1}%
	\begin{subequations}
		\renewcommand{\theparentequation}{#1}%
		\def\@currentlabel{#1}%
		}
		{%
	\end{subequations}\ignorespacesafterend
}
\makeatother


\newcommand{\ml}[1]{\begin{multlined}#1\end{multlined}}
\newcommand{\nn}{\nonumber\\ }


% Move down subscripts for some symbols like \chi
\NewDocumentCommand{\movedownsub}{e{^_}}{%
	\IfNoValueTF{#1}{%
		\IfNoValueF{#2}{^{}}% neither ^ nor _, do nothing; if no ^ but _, add ^{}
	}{%
		^{#1}% add superscript if present
	}%
	\IfNoValueF{#2}{_{#2}}% add subscript if present
}

% chi
\let\latexchi\chi
\RenewDocumentCommand{\chi}{}{\latexchi\movedownsub}


%Number sets
\newcommand{\Real}{\mathbb{R}}
\newcommand{\Nat}{\mathbb{N}}
\newcommand{\Rat}{\mathbb{Q}}
\newcommand{\Complex}{\mathbb{C}}

% imaginary number i
\newcommand{\iu}{\mathfrak{i}\mkern1mu}


% Calligraphic stuff
\newcommand{\calA}{\mathcal{A}}
\newcommand{\calB}{\mathcal{B}}
\newcommand{\calC}{\mathcal{C}}
\newcommand{\calD}{\mathcal{D}}
\newcommand{\calE}{\mathcal{E}}
\newcommand{\calF}{\mathcal{F}}
\newcommand{\calG}{\mathcal{G}}
\newcommand{\calH}{\mathcal{H}}
\newcommand{\calI}{\mathcal{I}}
\newcommand{\calJ}{\mathcal{J}}
\newcommand{\calK}{\mathcal{K}}
\newcommand{\calL}{\mathcal{L}}
\newcommand{\calM}{\mathcal{M}}
\newcommand{\calN}{\mathcal{N}}
\newcommand{\calO}{\mathcal{O}}
\newcommand{\calP}{\mathcal{P}}
\newcommand{\calQ}{\mathcal{Q}}
\newcommand{\calR}{\mathcal{R}}
\newcommand{\calS}{\mathcal{S}}
\newcommand{\calT}{\mathcal{T}}
\newcommand{\calU}{\mathcal{U}}
\newcommand{\calV}{\mathcal{V}}
\newcommand{\calW}{\mathcal{W}}
\newcommand{\calX}{\mathcal{X}}
\newcommand{\calY}{\mathcal{Y}}
\newcommand{\calZ}{\mathcal{Z}}

% Boldface stuff
\newcommand{\ba}{\mathbf{a}}
\newcommand{\bA}{\mathbf{A}}
\newcommand{\bb}{\mathbf{b}}
\newcommand{\bB}{\mathbf{B}}
\newcommand{\bc}{\mathbf{c}}
\newcommand{\bC}{\mathbf{C}}
\newcommand{\bd}{\mathbf{d}}
\newcommand{\bD}{\mathbf{D}}
\newcommand{\be}{\mathbf{e}}
\newcommand{\bE}{\mathbf{E}}
\newcommand{\boldf}{\mathbf{f}}
\newcommand{\bF}{\mathbf{F}}
\newcommand{\bg}{\mathbf{g}}
\newcommand{\bG}{\mathbf{G}}
\newcommand{\bh}{\mathbf{h}}
\newcommand{\bH}{\mathbf{H}}
\newcommand{\bi}{\mathbf{i}}
\newcommand{\bI}{\mathbf{I}}
\newcommand{\bj}{\mathbf{j}}
\newcommand{\bJ}{\mathbf{J}}
\newcommand{\bk}{\mathbf{k}}
\newcommand{\bK}{\mathbf{K}}
\newcommand{\bl}{\mathbf{l}}
\newcommand{\bL}{\mathbf{L}}
\newcommand{\boldm}{\mathbf{m}}
\newcommand{\bM}{\mathbf{M}}
\newcommand{\bn}{\mathbf{n}}
\newcommand{\bN}{\mathbf{N}}
\newcommand{\bo}{\mathbf{o}}
\newcommand{\bO}{\mathbf{O}}
\newcommand{\bp}{\mathbf{p}}
\newcommand{\bP}{\mathbf{P}}
\newcommand{\bq}{\mathbf{q}}
\newcommand{\bQ}{\mathbf{Q}}
\newcommand{\br}{\mathbf{r}}
\newcommand{\bR}{\mathbf{R}}
\newcommand{\bs}{\mathbf{s}}
\newcommand{\bS}{\mathbf{S}}
\newcommand{\bt}{\mathbf{t}}
\newcommand{\bT}{\mathbf{T}}
\newcommand{\bu}{\mathbf{u}}
\newcommand{\bU}{\mathbf{U}}
\newcommand{\bv}{\mathbf{v}}
\newcommand{\bV}{\mathbf{V}}
\newcommand{\bw}{\mathbf{w}}
\newcommand{\bW}{\mathbf{W}}
\newcommand{\bx}{\mathbf{x}}
\newcommand{\bX}{\mathbf{X}}
\newcommand{\by}{\mathbf{y}}
\newcommand{\bY}{\mathbf{Y}}
\newcommand{\bz}{\mathbf{z}}
\newcommand{\bZ}{\mathbf{Z}}


\newcommand{\mba}{\bm{a}}
\newcommand{\mbA}{\bm{A}}
\newcommand{\mbb}{\bm{b}}
\newcommand{\mbB}{\bm{B}}
\newcommand{\mbc}{\bm{c}}
\newcommand{\mbC}{\bm{C}}
\newcommand{\mbd}{\bm{d}}
\newcommand{\mbD}{\bm{D}}
\newcommand{\mbe}{\bm{e}}
\newcommand{\mbE}{\bm{E}}
\newcommand{\mbf}{\bm{f}}
\newcommand{\mbF}{\bm{F}}
\newcommand{\mbg}{\bm{g}}
\newcommand{\mbG}{\bm{G}}
\newcommand{\mbh}{\bm{h}}
\newcommand{\mbH}{\bm{H}}
\newcommand{\mbi}{\bm{i}}
\newcommand{\mbI}{\bm{I}}
\newcommand{\mbj}{\bm{j}}
\newcommand{\mbJ}{\bm{J}}
\newcommand{\mbk}{\bm{k}}
\newcommand{\mbK}{\bm{K}}
\newcommand{\mbl}{\bm{l}}
\newcommand{\mbL}{\bm{L}}
\newcommand{\mbm}{\bm{m}}
\newcommand{\mbM}{\bm{M}}
\newcommand{\mbn}{\bm{n}}
\newcommand{\mbN}{\bm{N}}
\newcommand{\mbo}{\bm{o}}
\newcommand{\mbO}{\bm{O}}
\newcommand{\mbp}{\bm{p}}
\newcommand{\mbP}{\bm{P}}
\newcommand{\mbq}{\bm{q}}
\newcommand{\mbQ}{\bm{Q}}
\newcommand{\mbr}{\bm{r}}
\newcommand{\mbR}{\bm{R}}
\newcommand{\mbs}{\bm{s}}
\newcommand{\mbS}{\bm{S}}
\newcommand{\mbt}{\bm{t}}
\newcommand{\mbT}{\bm{T}}
\newcommand{\mbu}{\bm{u}}
\newcommand{\mbU}{\bm{U}}
\newcommand{\mbv}{\bm{v}}
\newcommand{\mbV}{\bm{V}}
\newcommand{\mbw}{\bm{w}}
\newcommand{\mbW}{\bm{W}}
\newcommand{\mbx}{\bm{x}}
\newcommand{\mbX}{\bm{X}}
\newcommand{\mby}{\bm{y}}
\newcommand{\mbY}{\bm{Y}}
\newcommand{\mbz}{\bm{z}}
\newcommand{\mbZ}{\bm{Z}}

% Numbers bb font
\newcommand{\bbA}{\mathbb{A}}
\newcommand{\bbB}{\mathbb{B}}
\newcommand{\bbC}{\mathbb{C}}
\newcommand{\bbD}{\mathbb{D}}
\newcommand{\bbE}{\mathbb{E}}
\newcommand{\bbF}{\mathbb{F}}
\newcommand{\bbG}{\mathbb{G}}
\newcommand{\bbH}{\mathbb{H}}
\newcommand{\bbI}{\mathbb{I}}
\newcommand{\bbJ}{\mathbb{J}}
\newcommand{\bbK}{\mathbb{K}}
\newcommand{\bbL}{\mathbb{L}}
\newcommand{\bbM}{\mathbb{M}}
\newcommand{\bbN}{\mathbb{N}}
\newcommand{\bbO}{\mathbb{O}}
\newcommand{\bbP}{\mathbb{P}}
\newcommand{\bbQ}{\mathbb{Q}}
\newcommand{\bbR}{\mathbb{R}}
\newcommand{\bbS}{\mathbb{S}}
\newcommand{\bbT}{\mathbb{T}}
\newcommand{\bbU}{\mathbb{U}}
\newcommand{\bbV}{\mathbb{V}}
\newcommand{\bbW}{\mathbb{W}}
\newcommand{\bbX}{\mathbb{X}}
\newcommand{\bbY}{\mathbb{Y}}
\newcommand{\bbZ}{\mathbb{Z}}

% Mathfrak font
\newcommand{\frakA}{\mathfrak{A}}
\newcommand{\frakB}{\mathfrak{B}}
\newcommand{\frakC}{\mathfrak{C}}
\newcommand{\frakD}{\mathfrak{D}}
\newcommand{\frakE}{\mathfrak{E}}
\newcommand{\frakF}{\mathfrak{F}}
\newcommand{\frakG}{\mathfrak{G}}
\newcommand{\frakH}{\mathfrak{H}}
\newcommand{\frakI}{\mathfrak{I}}
\newcommand{\frakJ}{\mathfrak{J}}
\newcommand{\frakK}{\mathfrak{K}}
\newcommand{\frakL}{\mathfrak{L}}
\newcommand{\frakM}{\mathfrak{M}}
\newcommand{\frakN}{\mathfrak{N}}
\newcommand{\frakO}{\mathfrak{O}}
\newcommand{\frakP}{\mathfrak{P}}
\newcommand{\frakQ}{\mathfrak{Q}}
\newcommand{\frakR}{\mathfrak{R}}
\newcommand{\frakS}{\mathfrak{S}}
\newcommand{\frakT}{\mathfrak{T}}
\newcommand{\frakU}{\mathfrak{U}}
\newcommand{\frakV}{\mathfrak{V}}
\newcommand{\frakW}{\mathfrak{W}}
\newcommand{\frakX}{\mathfrak{X}}
\newcommand{\frakY}{\mathfrak{Y}}
\newcommand{\frakZ}{\mathfrak{Z}}

% Mathscr
\newcommand{\scA}{\mathscr{A}}
\newcommand{\scB}{\mathscr{B}}
\newcommand{\scC}{\mathscr{C}}
\newcommand{\scD}{\mathscr{D}}
\newcommand{\scE}{\mathscr{E}}
\newcommand{\scF}{\mathscr{F}}
\newcommand{\scG}{\mathscr{G}}
\newcommand{\scH}{\mathscr{H}}
\newcommand{\scI}{\mathscr{I}}
\newcommand{\scJ}{\mathscr{J}}
\newcommand{\scK}{\mathscr{K}}
\newcommand{\scL}{\mathscr{L}}
\newcommand{\scM}{\mathscr{M}}
\newcommand{\scN}{\mathscr{N}}
\newcommand{\scO}{\mathscr{O}}
\newcommand{\scP}{\mathscr{P}}
\newcommand{\scQ}{\mathscr{Q}}
\newcommand{\scR}{\mathscr{R}}
\newcommand{\scS}{\mathscr{S}}
\newcommand{\scT}{\mathscr{T}}
\newcommand{\scU}{\mathscr{U}}
\newcommand{\scV}{\mathscr{V}}
\newcommand{\scW}{\mathscr{W}}
\newcommand{\scX}{\mathscr{X}}
\newcommand{\scY}{\mathscr{Y}}
\newcommand{\scZ}{\mathscr{Z}}


% define some useful uppercase Greek letters in regular and bold sf
\DeclareSymbolFont{bsfletters}{OT1}{cmss}{bx}{n}
\DeclareSymbolFont{ssfletters}{OT1}{cmss}{m}{n}
\DeclareMathSymbol{\bsfGamma}{0}{bsfletters}{'000}
\DeclareMathSymbol{\ssfGamma}{0}{ssfletters}{'000}
\DeclareMathSymbol{\bsfDelta}{0}{bsfletters}{'001}
\DeclareMathSymbol{\ssfDelta}{0}{ssfletters}{'001}
\DeclareMathSymbol{\bsfTheta}{0}{bsfletters}{'002}
\DeclareMathSymbol{\ssfTheta}{0}{ssfletters}{'002}
\DeclareMathSymbol{\bsfLambda}{0}{bsfletters}{'003}
\DeclareMathSymbol{\ssfLambda}{0}{ssfletters}{'003}
\DeclareMathSymbol{\bsfXi}{0}{bsfletters}{'004}
\DeclareMathSymbol{\ssfXi}{0}{ssfletters}{'004}
\DeclareMathSymbol{\bsfPi}{0}{bsfletters}{'005}
\DeclareMathSymbol{\ssfPi}{0}{ssfletters}{'005}
\DeclareMathSymbol{\bsfSigma}{0}{bsfletters}{'006}
\DeclareMathSymbol{\ssfSigma}{0}{ssfletters}{'006}
\DeclareMathSymbol{\bsfUpsilon}{0}{bsfletters}{'007}
\DeclareMathSymbol{\ssfUpsilon}{0}{ssfletters}{'007}
\DeclareMathSymbol{\bsfPhi}{0}{bsfletters}{'010}
\DeclareMathSymbol{\ssfPhi}{0}{ssfletters}{'010}
\DeclareMathSymbol{\bsfPsi}{0}{bsfletters}{'011}
\DeclareMathSymbol{\ssfPsi}{0}{ssfletters}{'011}
\DeclareMathSymbol{\bsfOmega}{0}{bsfletters}{'012}
\DeclareMathSymbol{\ssfOmega}{0}{ssfletters}{'012}


% Greek
\newcommand{\balpha}{\bm{\alpha}}
\newcommand{\bbeta}{\bm{\beta}}
\newcommand{\bgamma}{\bm{\gamma}}
\newcommand{\bdelta}{\bm{\delta}}
\newcommand{\btheta}{\bm{\theta}}
\newcommand{\bmu}{\bm{\mu}}
\newcommand{\bnu}{\bm{\nu}}
\newcommand{\btau}{\bm{\tau}}
\newcommand{\bpi}{\bm{\pi}}
\newcommand{\bepsilon}{\bm{\epsilon}}
\newcommand{\bvarepsilon}{\bm{\varepsilon}}
\newcommand{\bsigma}{\bm{\sigma}}
\newcommand{\bvarsigma}{\bm{\varsigma}}
\newcommand{\bzeta}{\bm{\zeta}}
\newcommand{\bmeta}{\bm{\eta}}
\newcommand{\bkappa}{\bm{\kappa}}
\newcommand{\bchi}{\bm{\latexchi}\movedownsub}
\newcommand{\bphi}{\bm{\phi}}
\newcommand{\bpsi}{\bm{\psi}}
\newcommand{\bomega}{\bm{\omega}}
\newcommand{\bxi}{\bm{\xi}}
\newcommand{\blambda}{\bm{\lambda}}
\newcommand{\brho}{\bm{\rho}}

\newcommand{\bGamma}{\bm{\Gamma}}
\newcommand{\bLambda}{\bm{\Lambda}}
\newcommand{\bSigma	}{\bm{\Sigma}}
\newcommand{\bPsi}{\bm{\Psi}}
\newcommand{\bDelta}{\bm{\Delta}}
\newcommand{\bXi}{\bm{\Xi}}
\newcommand{\bUpsilon}{\bm{\Upsilon}}
\newcommand{\bOmega}{\bm{\Omega}}
\newcommand{\bPhi}{\bm{\Phi}}
\newcommand{\bPi}{\bm{\Pi}}
\newcommand{\bTheta}{\bm{\Theta}}

\newcommand{\talpha}{\widetilde{\alpha}}
\newcommand{\tbeta}{\widetilde{\beta}}
\newcommand{\tgamma}{\widetilde{\gamma}}
\newcommand{\tdelta}{\widetilde{\delta}}
\newcommand{\ttheta}{\widetilde{\theta}}
\newcommand{\tmu}{\widetilde{\mu}}
\newcommand{\tnu}{\widetilde{\nu}}
\newcommand{\ttau}{\widetilde{\tau}}
\newcommand{\tpi}{\widetilde{\pi}}
\newcommand{\tepsilon}{\widetilde{\epsilon}}
\newcommand{\tvarepsilon}{\widetilde{\varepsilon}}
\newcommand{\tsigma}{\widetilde{\sigma}}
\newcommand{\tvarsigma}{\widetilde{\varsigma}}
\newcommand{\tzeta}{\widetilde{\zeta}}
\newcommand{\tmeta}{\widetilde{\eta}}
\newcommand{\tkappa}{\widetilde{\kappa}}
\newcommand{\tchi}{\widetilde{\latexchi}\movedownsub}
\newcommand{\tphi}{\widetilde{\phi}}
\newcommand{\tpsi}{\widetilde{\psi}}
\newcommand{\tomega}{\widetilde{\omega}}
\newcommand{\txi}{\widetilde{\xi}}
\newcommand{\tlambda}{\widetilde{\lambda}}
\newcommand{\trho}{\widetilde{\rho}}

\newcommand{\tGamma}{\widetilde{\Gamma}}
\newcommand{\tDelta}{\widetilde{\Delta}}
\newcommand{\tTheta}{\widetilde{\Theta}}
\newcommand{\tPi}{\widetilde{\Pi}}
\newcommand{\tSigma}{\widetilde{\Sigma}}
\newcommand{\tPhi}{\widetilde{\Phi}}
\newcommand{\tPsi}{\widetilde{\Psi}}
\newcommand{\tOmega}{\widetilde{\Omega}}
\newcommand{\tXi}{\widetilde{\Xi}}
\newcommand{\tLambda}{\widetilde{\Lambda}}

\newcommand{\tbalpha}{\widetilde{\balpha}}
\newcommand{\tbbeta}{\widetilde{\bbeta}}
\newcommand{\tbgamma}{\widetilde{\bgamma}}
\newcommand{\tbdelta}{\widetilde{\bdelta}}
\newcommand{\tbtheta}{\widetilde{\btheta}}
\newcommand{\tbmu}{\widetilde{\bmu}}
\newcommand{\tbnu}{\widetilde{\bnu}}
\newcommand{\tbtau}{\widetilde{\btbau}}
\newcommand{\tbpi}{\widetilde{\bpi}}
\newcommand{\tbepsilon}{\widetilde{\bepsilon}}
\newcommand{\tbvarepsilon}{\widetilde{\bvarepsilon}}
\newcommand{\tbsigma}{\widetilde{\bsigma}}
\newcommand{\tbvarsigma}{\widetilde{\bvarsigma}}
\newcommand{\tbzeta}{\widetilde{\bzeta}}
\newcommand{\tbmeta}{\widetilde{\beta}}
\newcommand{\tbkappa}{\widetilde{\bkappa}}
\newcommand{\tbchi}{\widetilde\bm{\latexchi}\movedownsub}
\newcommand{\tbphi}{\widetilde{\bphi}}
\newcommand{\tbpsi}{\widetilde{\bpsi}}
\newcommand{\tbomega}{\widetilde{\bomega}}
\newcommand{\tbxi}{\widetilde{\bxi}}
\newcommand{\tblambda}{\widetilde{\blambda}}
\newcommand{\tbrho}{\widetilde{\brho}}

\newcommand{\tbGamma}{\widetilde{\bGamma}}
\newcommand{\tbDelta}{\widetilde{\bDelta}}
\newcommand{\tbTheta}{\widetilde{\bTheta}}
\newcommand{\tbPi}{\widetilde{\bPi}}
\newcommand{\tbSigma}{\widetilde{\bSigma}}
\newcommand{\tbPhi}{\widetilde{\bPhi}}
\newcommand{\tbPsi}{\widetilde{\bPsi}}
\newcommand{\tbOmega}{\widetilde{\bOmega}}
\newcommand{\tbXi}{\widetilde{\bXi}}
\newcommand{\tbLambda}{\widetilde{\bLambda}}

\newcommand{\halpha}{\widehat{\alpha}}
\newcommand{\hbeta}{\widehat{\beta}}
\newcommand{\hgamma}{\widehat{\gamma}}
\newcommand{\hdelta}{\widehat{\delta}}
\newcommand{\htheta}{\widehat{\theta}}
\newcommand{\hmu}{\widehat{\mu}}
\newcommand{\hnu}{\widehat{\nu}}
\newcommand{\htau}{\widehat{\tau}}
\newcommand{\hpi}{\widehat{\pi}}
\newcommand{\hepsilon}{\widehat{\epsilon}}
\newcommand{\hvarepsilon}{\widehat{\varepsilon}}
\newcommand{\hsigma}{\widehat{\sigma}}
\newcommand{\hvarsigma}{\widehat{\varsigma}}
\newcommand{\hzeta}{\widehat{\zeta}}
\newcommand{\hmeta}{\widehat{\eta}}
\newcommand{\hkappa}{\widehat{\kappa}}
\newcommand{\hchi}{\widehat{\latexchi}\movedownsub}
\newcommand{\hphi}{\widehat{\phi}}
\newcommand{\hpsi}{\widehat{\psi}}
\newcommand{\homega}{\widehat{\omega}}
\newcommand{\hxi}{\widehat{\xi}}
\newcommand{\hlambda}{\widehat{\lambda}}
\newcommand{\hrho}{\widehat{\rho}}

\newcommand{\hGamma}{\widehat{\Gamma}}
\newcommand{\hDelta}{\widehat{\Delta}}
\newcommand{\hTheta}{\widehat{\Theta}}
\newcommand{\hPi}{\widehat{\Pi}}
\newcommand{\hSigma}{\widehat{\Sigma}}
\newcommand{\hPhi}{\widehat{\Phi}}
\newcommand{\hPsi}{\widehat{\Psi}}
\newcommand{\hOmega}{\widehat{\Omega}}
\newcommand{\hXi}{\widehat{\Xi}}
\newcommand{\hLambda}{\widehat{\Lambda}}

\newcommand{\hbalpha}{\widehat{\balpha}}
\newcommand{\hbbeta}{\widehat{\bbeta}}
\newcommand{\hbgamma}{\widehat{\bgamma}}
\newcommand{\hbdelta}{\widehat{\bdelta}}
\newcommand{\hbtheta}{\widehat{\btheta}}
\newcommand{\hbmu}{\widehat{\bmu}}
\newcommand{\hbnu}{\widehat{\bnu}}
\newcommand{\hbtau}{\widehat{\btau}}
\newcommand{\hbpi}{\widehat{\bpi}}
\newcommand{\hbepsilon}{\widehat{\bepsilon}}
\newcommand{\hbvarepsilon}{\widehat{\bvarepsilon}}
\newcommand{\hbsigma}{\widehat{\bsigma}}
\newcommand{\hbvarsigma}{\widehat{\bvarsigma}}
\newcommand{\hbzeta}{\widehat{\bzeta}}
\newcommand{\hbmeta}{\widehat{\beta}}
\newcommand{\hbkappa}{\widehat{\bkappa}}
\newcommand{\hbchi}{\widehat\bm{\latexchi}\movedownsub}
\newcommand{\hbphi}{\widehat{\bphi}}
\newcommand{\hbpsi}{\widehat{\bpsi}}
\newcommand{\hbomega}{\widehat{\bomega}}
\newcommand{\hbxi}{\widehat{\bxi}}
\newcommand{\hblambda}{\widehat{\blambda}}
\newcommand{\hbrho}{\widehat{\brho}}

\newcommand{\hbGamma}{\widehat{\bGamma}}
\newcommand{\hbDelta}{\widehat{\bDelta}}
\newcommand{\hbTheta}{\widehat{\bTheta}}
\newcommand{\hbPi}{\widehat{\bPi}}
\newcommand{\hbSigma}{\widehat{\bSigma}}
\newcommand{\hbPhi}{\widehat{\bPhi}}
\newcommand{\hbPsi}{\widehat{\bPsi}}
\newcommand{\hbOmega}{\widehat{\bOmega}}
\newcommand{\hbXi}{\widehat{\bXi}}
\newcommand{\hbLambda}{\widehat{\bLambda}}

\makeatletter
\newcommand*\rel@kern[1]{\kern#1\dimexpr\macc@kerna}
\newcommand*\widebar[1]{%
  \begingroup
  \def\mathaccent##1##2{%
    \rel@kern{0.8}%
    \overline{\rel@kern{-0.8}\macc@nucleus\rel@kern{0.2}}%
    \rel@kern{-0.2}%
  }%
  \macc@depth\@ne
  \let\math@bgroup\@empty \let\math@egroup\macc@set@skewchar
  \mathsurround\z@ \frozen@everymath{\mathgroup\macc@group\relax}%
  \macc@set@skewchar\relax
  \let\mathaccentV\macc@nested@a
  \macc@nested@a\relax111{#1}%
  \endgroup
}
\makeatother

\newcommand{\barbalpha}{\widebar{\balpha}}
\newcommand{\barbbeta}{\widebar{\bbeta}}
\newcommand{\barbgamma}{\widebar{\bgamma}}
\newcommand{\barbdelta}{\widebar{\bdelta}}
\newcommand{\barbtheta}{\widebar{\btheta}}
\newcommand{\barbmu}{\widebar{\bmu}}
\newcommand{\barbnu}{\widebar{\bnu}}
\newcommand{\barbtau}{\widebar{\btau}}
\newcommand{\barbpi}{\widebar{\bpi}}
\newcommand{\barbepsilon}{\widebar{\bepsilon}}
\newcommand{\barbvarepsilon}{\widebar{\bvarepsilon}}
\newcommand{\barbsigma}{\widebar{\bsigma}}
\newcommand{\barbvarsigma}{\widebar{\bvarsigma}}
\newcommand{\barbzeta}{\widebar{\bzeta}}
\newcommand{\barbmeta}{\widebar{\beta}}
\newcommand{\barbkappa}{\widebar{\bkappa}}
\newcommand{\barbchi}{\bar\bm{\latexchi}\movedownsub}
\newcommand{\barbphi}{\widebar{\bphi}}
\newcommand{\barbpsi}{\widebar{\bpsi}}
\newcommand{\barbomega}{\widebar{\bomega}}
\newcommand{\barbxi}{\widebar{\bxi}}
\newcommand{\barblambda}{\widebar{\blambda}}
\newcommand{\barbrho}{\widebar{\brho}}

\newcommand{\barbGamma}{\widebar{\bGamma}}
\newcommand{\barbDelta}{\widebar{\bDelta}}
\newcommand{\barbTheta}{\widebar{\bTheta}}
\newcommand{\barbPi}{\widebar{\bPi}}
\newcommand{\barbSigma}{\widebar{\bSigma}}
\newcommand{\barbPhi}{\widebar{\bPhi}}
\newcommand{\barbPsi}{\widebar{\bPsi}}
\newcommand{\barbOmega}{\widebar{\bOmega}}
\newcommand{\barbXi}{\widebar{\bXi}}
\newcommand{\barbLambda}{\widebar{\bLambda}}

%MathOperator
\DeclareMathOperator*{\argmax}{arg\,max}
\DeclareMathOperator*{\argmin}{arg\,min}
\DeclareMathOperator*{\argsup}{arg\,sup}
\DeclareMathOperator*{\arginf}{arg\,inf}
\DeclareMathOperator*{\minimize}{minimize}
\DeclareMathOperator*{\maximize}{maximize}
\DeclareMathOperator{\st}{s.t.\ }
%\DeclareMathOperator{\st}{subject\,\,to}
\DeclareMathOperator{\as}{a.s.}
\DeclareMathOperator{\const}{const}
\DeclareMathOperator{\diag}{diag}
\DeclareMathOperator{\cum}{cum}
\DeclareMathOperator{\sgn}{sgn}
\DeclareMathOperator{\tr}{tr}
\DeclareMathOperator{\Tr}{Tr}
\DeclareMathOperator{\spn}{span}
\DeclareMathOperator{\supp}{supp}
\DeclareMathOperator{\adj}{adj}
\DeclareMathOperator{\var}{var}
\DeclareMathOperator{\Vol}{Vol}
\DeclareMathOperator{\cov}{cov}
\DeclareMathOperator{\corr}{corr}
\DeclareMathOperator{\sech}{sech}
\DeclareMathOperator{\sinc}{sinc}
\DeclareMathOperator{\rank}{rank}
\DeclareMathOperator{\poly}{poly}
\DeclareMathOperator{\vect}{vec}
\DeclareMathOperator{\conv}{conv}
\DeclareMathOperator*{\lms}{l.i.m.\,}
\DeclareMathOperator*{\esssup}{ess\,sup}
\DeclareMathOperator*{\essinf}{ess\,inf}
\DeclareMathOperator{\sign}{sign}
\DeclareMathOperator{\eig}{eig}
\DeclareMathOperator{\ima}{im}
\DeclareMathOperator{\Mod}{mod}

%Paired delimiters
\newcommand{\ifbcdot}[1]{\ifblank{#1}{\cdot}{#1}}

\DeclarePairedDelimiterX\abs[1]{\lvert}{\rvert}{\ifbcdot{#1}}
\DeclarePairedDelimiterX\parens[1]{(}{)}{\ifbcdot{#1}}
\DeclarePairedDelimiterX\brk[1]{[}{]}{\ifbcdot{#1}}
\DeclarePairedDelimiterX\braces[1]{\{}{\}}{\ifbcdot{#1}}
\DeclarePairedDelimiterX\angles[1]{\langle}{\rangle}{#1}
\DeclarePairedDelimiterX\ip[2]{\langle}{\rangle}{\ifbcdot{#1},\ifbcdot{#2}}
\DeclarePairedDelimiterX\norm[1]{\lVert}{\rVert}{\ifbcdot{#1}}
\DeclarePairedDelimiterX\ceil[1]{\lceil}{\rceil}{\ifbcdot{#1}}
\DeclarePairedDelimiterX\floor[1]{\lfloor}{\rfloor}{\ifbcdot{#1}}

\DeclarePairedDelimiterXPP\trace[1]{\operatorname{Tr}}{(}{)}{}{\ifbcdot{#1}} % column vector
\DeclarePairedDelimiterXPP\col[1]{\operatorname{col}}{\{}{\}}{}{\ifbcdot{#1}} % column vector
\DeclarePairedDelimiterXPP\row[1]{\operatorname{row}}{\{}{\}}{}{\ifbcdot{#1}} % row vector
\DeclarePairedDelimiterXPP\erf[1]{\operatorname{erf}}{(}{)}{}{\ifbcdot{#1}}
\DeclarePairedDelimiterXPP\erfc[1]{\operatorname{erfc}}{(}{)}{}{\ifbcdot{#1}}
\DeclarePairedDelimiterXPP\KLD[2]{D}{(}{)}{}{\ifbcdot{#1}\, \delimsize\|\, \ifbcdot{#2}} % KL divergence
\DeclarePairedDelimiterXPP\op[2]{\operatorname{#1}}{(}{)}{}{#2} % general operator

% Math relations
\newcommand{\convp}{\stackrel{\mathrm{p}}{\longrightarrow}}
\newcommand{\convas}{\stackrel{\mathrm{a.s.}}{\longrightarrow}}
\newcommand{\convd}{\stackrel{\mathrm{d}}{\longrightarrow}}
\newcommand{\convD}{\stackrel{\mathrm{D}}{\longrightarrow}}

\newcommand{\dotleq}{\stackrel{.}{\leq}}
\newcommand{\dotlt}{\stackrel{.}{<}}
\newcommand{\dotgeq}{\stackrel{.}{\geq}}
\newcommand{\dotgt}{\stackrel{.}{>}}
\newcommand{\dotdoteq}{\stackrel{\,..}{=}}

\newcommand{\eqa}[1]{\stackrel{#1}{=}}
\newcommand{\ed}{\eqa{\mathrm{d}}}
\newcommand{\lea}[1]{\stackrel{#1}{\le}}
\newcommand{\gea}[1]{\stackrel{#1}{\ge}}

\newcommand{\T}{^{\intercal}}% transpose notation
\newcommand{\setcomp}{^{\mathsf{c}}} %set complement
\newcommand{\ud}{\,\mathrm{d}} % for integrals like \int f(x) \ud x
\newcommand{\Id}{\mathrm{Id}} % identity function
\newcommand{\Bigmid}{{\ \Big| \ }}
\newcommand{\bzero}{\bm{0}}
\newcommand{\bone}{\bm{1}}

% Math functions
\newcommand{\indicator}[1]{{\bf 1}_{\braces*{#1}}}
\newcommand{\indicatore}[1]{{\bf 1}_{#1}}
\newcommand{\indicate}[1]{{\bf 1}\braces*{#1}}
\newcommand{\ofrac}[1]{{\frac{1}{#1}}}
\newcommand{\odfrac}[1]{{\dfrac{1}{#1}}}
\newcommand{\ddfrac}[2]{{\dfrac{\mathrm{d} {#1}}{\mathrm{d} {#2}}}}
\newcommand{\ppfrac}[2]{\dfrac{\partial {#1}}{\partial {#2}}}
\newcommand{\tc}[1]{^{(#1)}}

\newcommand{\bmat}[1]{\begin{bmatrix} #1 \end{bmatrix}}
\newcommand{\smat}[1]{\left[\begin{smallmatrix} #1 \end{smallmatrix}\right]}

\newcommand{\Lh}[1]{\ell_{#1}}
\newcommand{\LLh}[1]{\log{\Lh{#1}}}

% just to make sure it exists
\providecommand\given{}
% can be useful to refer to this outside \set
\newcommand\SetSymbol[2][]{%
	\nonscript\, #1#2
	\allowbreak
	\nonscript\,
	\mathopen{}}

\DeclarePairedDelimiterX\Set[2]\{\}{%
\renewcommand\given{\SetSymbol[\delimsize]{#1}}
#2
}
\DeclarePairedDelimiterX\Setc[1]\{\}{%
\renewcommand\given{\SetSymbol{:}}
#1
}

% \set{x \given f(x)=1} gives \{x : f(x)=1\}
% \set[\vert]{x \given f(x)=1} gives \{x \vert f(x)=1\}
% Starred version uses \left and \right
\NewDocumentCommand\set{s o m}{%
	\IfBooleanTF#1%
	{\IfValueTF{#2}{\Set*{#2}{#3}}{\Setc*{#3}}}%
	{\IfValueTF{#2}{\Set{#2}{#3}}{\Setc{#3}}}%
}

%\NewDocumentCommand\set{s m t| m}{%
%\IfBooleanTF#1%
%{\left\{\, #2\mathrel{} \IfBooleanTF{#3}{\middle|}{:}\mathrel{}  #4\, \right\}}%
%{\{\, #2 \IfBooleanTF{#3}{\mid}{\mathrel{} : \mathrel{}} #4\, \}}% 
%}

\NewDocumentCommand{\evalat}{s O{\big} m m}{%
\IfBooleanTF{#1}%
{{\left. #3 \right|_{#4}}}
{{#3#2|_{#4}}}%
}


\NewDocumentCommand \ifcondp {m m} {%
	#1%
	\IfValueT{#2}{\given #2}%
}

\providecommand\given{}
\DeclarePairedDelimiterXPP\cprob[1]{}(){}{
\renewcommand\given{\nonscript\,\delimsize\vert\allowbreak\nonscript\,\mathopen{}}
\ifcondp#1
}
\DeclarePairedDelimiterXPP\cexp[1]{}[]{}{
\renewcommand\given{\nonscript\,\delimsize\vert\allowbreak\nonscript\,\mathopen{}}
\ifcondp#1
}


% Allows the use of 
% \P : \mathbb{P}
% \P(X) : \mathbb{P}\left({X}\right)
% \P_{p}(X) : \mathbb{P}_{p}\left({X}\right)
% \P(X \given Y) or \P(X @| Y) or \P(X){Y} : \mathbb{P}\left({X}\, \middle| \, {Y}\right). 
% \P_{p}(X \given Y) or \P_{p}(X @| Y) : \mathbb{P}_{p}\left({X}\, \middle| \, {Y}\right)
% Caveats: Iterated expressions do not work well with \P(X @| Y) notation
% \P(\P(X @| Y) @| Z) does not work, use \P(\P(X \given Y) \given Z) 
% Starred version \P* does not use \left and \right. Maybe used in inline equations. 
\DeclareDocumentCommand \P { s e{_} >{\SplitArgument{ 1 }{ @| }}d() g } {%
	\mathbb{P}%
	\IfBooleanTF{#1}%
		{
			\IfValueT{#2}{_{#2}}%
			\IfValueTF{#4}%
				{\cprob{#3 \given #4}}%
				{\IfValueT{#3}{\cprob{#3}}}%
		}%
		{
			\IfValueT{#2}{_{#2}}%
			\IfValueTF{#4}%
				{\cprob*{#3 \given #4}}%
				{\IfValueT{#3}{\cprob*{#3}}}%
		}%
}

% Allows the use of 
% \E : \mathbb{E}
% \E[X] : \mathbb{E}\left[{X}\right]
% \E_{p}[X] or \E{p}[X] : \mathbb{E}_{p}\left[{X}\right]
% \E[X \given Y] or \E[X @| Y] or \E[X]{Y} : \mathbb{E}\left[{X}\, \middle| \, {Y}\right]. 
% \E_{p}[X \given Y] or \E_{p}[X @| Y] : \mathbb{E}_{p}\left[{X}\, \middle| \, {Y}\right]
% Caveats: Iterated expressions do not work well with \E[X @| Y] notation
% \E[\E[X @| Y] @| Z] does not work, use \E[\E[X \given Y] \given Z] 
% Starred version \E* does not use \left and \right. Maybe used in inline equations. 
\DeclareDocumentCommand \E { s e{_} >{\SplitArgument{ 1 }{ @| }}d[] g } {%
	\mathbb{E}%
	\IfBooleanTF{#1}%
		{
			\IfValueT{#2}{_{#2}}%
			\IfValueTF{#4}%
				{\cexp{#3 \given #4}}%
				{\IfValueT{#3}{\cexp{#3}}}%
		}%
		{
			\IfValueT{#2}{_{#2}}%
			\IfValueTF{#4}%
				{\cexp*{#3 \given #4}}%
				{\IfValueT{#3}{\cexp*{#3}}}%
		}%
}

% General distribution 
% E.g., \dist{Beta}[a,b][x] gives Beta(x | a,b); \dist{Beta}[a,b] gives Beta(a,b)
\ExplSyntaxOn
\NewDocumentCommand \dist {m o o} {%
\mathrm{#1}\left(%
	\IfValueT{#3}{%
		\tl_if_blank:nTF{ #3 }{\cdot\, \middle|\, }{#3\, \middle|\, }%
	}
	\IfValueT{#2}{#2}%
\right)%
}
\ExplSyntaxOff

\newcommand{\Bern}[1]{\dist{Bern}[#1]}
\newcommand{\Unif}[1]{\dist{Unif}[#1]}
\newcommand{\Dir}[1]{\dist{Dir}[#1]}
\newcommand{\Cat}[1]{\dist{Cat}[#1]}
\newcommand{\N}[2]{\dist{\calN}[#1,\, #2]}
\newcommand{\Beta}[2]{\dist{Beta}[#1,\, #2]}

\def\indep#1#2{\mathrel{\rlap{$#1#2$}\mkern5mu{#1#2}}}
\newcommand{\independent}{\protect\mathpalette{\protect\indep}{\perp}}


%Misc

% Colored underbrace/overbrace
\NewDocumentCommand {\cbrace} { D[]{black} d[] D(){\widthof{#5}} m m } {%
	\begingroup%
		\color{#1}
		\IfValueTF{#2}{%
			\overbrace{{\color{#1}#4}}^%
		}{
			\underbrace{#4}_%
		}%
		{\parbox[c]{#3}{\centering\footnotesize{#5}}}%
	\endgroup% 
}


\let\oldforall\forall
\renewcommand{\forall}{\oldforall \, }

\let\oldexist\exists
\renewcommand{\exists}{\oldexist \, }

\newcommand\existu{\oldexist! \, }


% Figures
\renewcommand{\figurename}{Fig.}
\newcommand{\figref}[1]{\figurename~\ref{#1}}
\graphicspath{{./Figures/}{./figures/}}
\pdfsuppresswarningpagegroup=1

\newcommand{\includeCroppedPdf}[2][]{%
	\IfFileExists{./Figures/#2-crop.pdf}{}{%
		\immediate\write18{pdfcrop ./Figures/#2 ./Figures/#2-crop.pdf}}%
	\includegraphics[#1]{./Figures/#2-crop.pdf}}


%%%%%%%%%%%%%%%%%%%%%%%%%%%%%%%%%%%%%%%%%%%%%%%%%%%%%%%%%%%%%%%%%%%%%%%%%

% Supplement
\newcommand{\beginsupplement}{
	\setcounter{section}{0}
	\renewcommand{\thesection}{S\arabic{section}}
	\setcounter{equation}{0}
	\renewcommand{\theequation}{S\arabic{equation}}
	\setcounter{table}{0}
	\renewcommand{\thetable}{S\arabic{table}}
	\setcounter{figure}{0}
	\renewcommand{\thefigure}{S\arabic{figure}}
}


% Editing
\definecolor{gray90}{gray}{0.9}

\ifx\nohighlights\undefined
	\newcommand{\red}[1]{{\color{red} #1}}
	\newcommand{\blue}[1]{{{\color{blue} #1}}}
	\newcommand{\msout}[1]{\text{\color{green} \sout{\ensuremath{#1}}}}
	\newcommand{\del}[1]{{\color{green}\ifmmode \msout{#1}\else\sout{#1}\fi}}
\else
	\newcommand{\red}[1]{#1}
	\newcommand{\blue}[1]{#1}
	\newcommand{\msout}[1]{#1}
	\newcommand{\del}[1]{#1}
\fi

\newcommand{\old}[1]{{\color{green} [\textrm{DELETED: }#1]}}
\newcommand{\hhide}[1]{}
%\newcommand{\hhide}[1]{{\color{magenta} [TO BE EXCLUDED] #1}}

\newcommand{\txp}[2]{\texorpdfstring{#1}{#2}}

%%%%%%%%%%%%%%%%%%%%%%%%%%%%%%%%%%%%%%%%%%%%%%%%%
% For diagnosis: if activated, will show what is causing 
% LaTeX Warning: Label(s) may have changed. Rerun to get cross-references right.

\ifx\diagnoselabel\undefined
	\relax
\else
	\makeatletter
	\def\@testdef #1#2#3{%
		\def\reserved@a{#3}\expandafter \ifx \csname #1@#2\endcsname
			\reserved@a  \else
			\typeout{^^Jlabel #2 changed:^^J%
				\meaning\reserved@a^^J%
				\expandafter\meaning\csname #1@#2\endcsname^^J}%
			\@tempswatrue \fi}
	\makeatother
\fi

%%%%%%%%%%%%%%%%%%%%%%%%%%%%%%%%%%%%%%%%%%%%%%%%%%

\def\UrlFont{\tt}


\newcounter{week}

%Theorem declarations
\newtheorem{Theorem}{Theorem}[week]
\newtheorem{Corollary}{Corollary}[week]
\newtheorem{Proposition}{Proposition}[week]
\newtheorem{Lemma}{Lemma}[week]
\newtheorem{Definition}{Definition}[week]
\newtheorem{Assumption}{Assumption}[week]

%\theoremstyle{definition}
\newtheorem{Example}{Example}[week]
\newtheorem{Remark}{Remark}[week]
\newtheorem{Exercise}{Exercise}[week]


\newcommand{\handout}[2]{
	\setcounter{week}{#1}
  \noindent
  \begin{center}
  \framebox{
    \vbox{
      \hbox to 6in {\bf An Analytical Introduction to Probability Theory \hfill}
      \vspace{5mm}
      \hbox to 6in { {\Large \hfill #1.~#2  \hfill} }
      \vspace{5mm}
      \hbox to 6in { {\em SIGNAL, NTU \hfill \small{\url{https://personal.ntu.edu.sg/wptay/}}}}
    }
  }
  \end{center}
  \setcounter{section}{0}
	\renewcommand{\thesection}{{#1}.\arabic{section}}
  \setcounter{Theorem}{0}
  \setcounter{Corollary}{0}
  \setcounter{Proposition}{0}
  \setcounter{Lemma}{0}
  \setcounter{Definition}{0}
  \setcounter{Assumption}{0}
  \setcounter{Example}{0}
  \setcounter{Remark}{0}
  \setcounter{Exercise}{0}
  \vspace*{4mm}
}

\newcommand{\calBR}{\calB(\Real)}
\newcommand{\io}{\ \mathrm{i.o.}}
\newcommand{\fo}{\ \mathrm{f.o.}}
\newcommand{\Po}[1]{\mathrm{Po}\left(#1\right)}
\externaldocument{1_BasicRealAnalysis_I}
\externaldocument{3_ProbabilitySpaces}
\externaldocument{4_Random_Variables}
\externaldocument{5_Convergence_and_Independence}
\externaldocument{6_Borel_Cantelli_Lemmas}
\externaldocument{7_StrongLaw_of_LargeNumbers}
\externaldocument{8_Glivenko-Cantelli_and_0-1_Laws}
\externaldocument{9_Weak_Convergence}
\externaldocument{10_Characteristic_Functions}

\newcommand{\Po}[1]{\mathrm{Po}\left(#1\right)}

\begin{document}

\handout{11}{Lindeberg's Central Limit Theorem}

\section{Lindeberg's Method}
Recall the CLT for an i.i.d.\ sequence from the last session: Let $X_1, X_2, \ldots$ be an i.i.d.\ sequence with $\E X_i = 0$ and $\var X_i = 1$. We have
\begin{align*}
\frac{S_n}{\sqrt{n}} \convd \calN (0, 1).
\end{align*}
In this session, we show another proof of this CLT using Lindeberg's method. Let $G_1, G_2, \ldots$ be i.i.d. $\calN(0, 1)$, independent of $X_1,X_2,\ldots$. Let
\begin{align*}
T_{m,n} = \frac{1}{\sqrt{n}} (G_1 + \ldots + G_{m-1} + X_m + \ldots + X_n)
\end{align*}
with
\begin{align*}
T_{n+1,n} =& \frac{1}{\sqrt{n}} (G_1 + \ldots + G_n) \sim \calN (0, 1), \\
T_{1,n} =& \frac{1}{\sqrt{n}} (X_1 + \ldots + X_n).
\end{align*}
We aim to show that $T_{1,n} \convd T_{n+1,n}$ as $n \to \infty$. Let $f \in C_b(\Real)$ with $c_2 = \sup \abs*{f\tc{2}} < \infty$ and $c_3 = \sup \abs*{f\tc{3}} < \infty$. We have
\begin{align*}
&\abs*{\E f(T_{1,n}) - \E f(T_{n+1,n})} \\
&= \abs*{ \sum_{m=1}^{n} \left( \E f(T_{m,n}) - \E f(T_{m+1,n}) \right)} \\
&\leq \sum_{m=1}^{n} \abs*{ \E f(T_{m,n}) - \E f(T_{m+1,n}) }.
\end{align*}
Let
\begin{align*}
U_m = \frac{1}{\sqrt{n}} \left(G_1 + \ldots + G_{m-1} + X_{m+1} + \ldots + X_n\right).
\end{align*}
Then,
\begin{align*}
T_{m,n} = U_m + \frac{X_m}{\sqrt{n}}, \\
T_{m+1,n} = U_m + \frac{G_m}{\sqrt{n}}.
\end{align*}
Using Taylor series expansion at $U_m$ (see \cref{wk10:sec:CLT_IID}), for any $\epsilon>0$, we obtain
\begin{align*}
\abs*{f(T_{m,n}) - f(U_m) - f'(U_m)\frac{X_m}{\sqrt{n}} - f''(U_m)\frac{X_m^2}{2n}} \indicator{\abs*{X_m} \leq \epsilon \sqrt{n}} 
\leq&
\frac{c_3\abs*{X_m}^3}{6n^{3/2}} \indicator{\abs*{X_m} \leq \epsilon \sqrt{n}}
\leq
\frac{c_3\epsilon}{6n} X_m^2, \\
%
\abs*{f(T_{m,n}) - f(U_m) - f'(U_m)\frac{X_m}{\sqrt{n}} } \indicator{\abs*{X_m} > \epsilon \sqrt{n}}
\leq&
\frac{c_2 X_m^2}{2n} \indicator{\abs*{X_m} > \epsilon \sqrt{n}}.
\end{align*}

By combining the inequality $\abs*{f''(U_m)\frac{X_m^2}{2n}} \leq \frac{c_2 X_m^2}{2n}$ with the second inequality above, we obtain
\begin{align*}
\abs*{f(T_{m,n}) - f(U_m) - f'(U_m)\frac{X_m}{\sqrt{n}} - f''(U_m)\frac{X_m^2}{2n}} \indicator{\abs*{X_m} > \epsilon \sqrt{n}}
\leq
\frac{c_2 X_m^2}{n} \indicator{\abs*{X_m} > \epsilon \sqrt{n}}.
\end{align*}
Therefore, we have
\begin{align*}
\abs*{f(T_{m,n}) - f(U_m) - f'(U_m)\frac{X_m}{\sqrt{n}} - f''(U_m)\frac{X_m^2}{2n}}
\leq
\frac{c_3\epsilon}{6n}X_m^2 + \frac{c_2\ X_m^2}{n} \indicator{\abs*{X_m} > \epsilon \sqrt{n}}.
\end{align*}
Similarly, we can obtain
\begin{align*}
\abs*{f(T_{m+1,n}) - f(U_m) - f'(U_m)\frac{G_m}{\sqrt{n}} - f''(U_m)\frac{G_m^2}{2n}}
\leq
\frac{c_3\epsilon}{6n}G_m^2 + \frac{c_2 G_m^2}{n} \indicator{\abs*{G_m} > \epsilon \sqrt{n}}.
\end{align*}
Furthermore,  by definition, we have
\begin{align*}
\E[f'(U_m)X_m] =& \E[f'(U_m)] \E[X_m] = 0, \\
\E[f''(U_m)X_m^2] =& \E[f''(U_m)] \E[X_m^2] = \E[f''(U_m)] = \E[f''(U_m)G_m^2],
\end{align*}
and
\begin{align*}
\abs*{ \E[f(T_{m,n})] - \E[f(T_{m+1,n})] } 
\leq
\frac{c_3\epsilon}{3n} + \frac{c_2}{n} \left( \E X_m^2 \indicator{\abs*{X_m} > \epsilon \sqrt{n}} +  \E G_m^2 \indicator{\abs*{G_m} > \epsilon \sqrt{n}} \right).
\end{align*}
Summing over $1\leq m \leq n$, we have
\begin{align*}
\abs*{\E[f(T_{1,n})] - \E[f(T_{n+1,n})]}
\leq
\frac{c_3\epsilon}{3} + c_2\left( \E X_1^2 \indicator{\abs*{X_1} > \epsilon \sqrt{n}} +  \E G_1^2 \indicator{\abs*{G_1} > \epsilon \sqrt{n}} \right).
\end{align*}
By the DCT (\cref{Dominated Convergence Theorem}), we obtain
\begin{align*}
&\E X_1^2 \indicator{\abs{X_1} > \epsilon \sqrt{n}} \to \E X_1^2 \indicator{\abs{X_1} = \infty} =0,\\
&\E G_1^2 \indicator{\abs{G_1} > \epsilon \sqrt{n}} \to \E G_1^2 \indicator{\abs{G_1} = \infty} =0,
\end{align*}
as $n \to \infty$. Therefore, we have $\abs{\E[f(T_{1,n})] - \E[f(T_{n+1,n})]}\to 0$ by taking $n\to\infty$ and $\epsilon\to0$.


\section{Lindeberg's CLT}
\begin{Theorem}[Lindeberg's CLT]
For each $n \geq 1$, let $\left(X_{n,m}\right)_{m=1}^n$ be a sequence of independent random variables with $\E X_{n,m} = 0$. Then, $S_n = \sum_{m=1}^n X_{n,m} \convd \calN (0,1) \ \text{as} \ n \to \infty$ if
\begin{enumerate}[(i)]
\item\label[condition]{Lindeberg_condition1} $\sum_{m=1}^n \E X_{n,m}^2 \to 1$ as $n \to \infty$; and
\item\label[condition]{Lindeberg_condition2} $\forall \epsilon > 0$, $\sum_{m=1}^n \E X_{n,m}^2 \indicator{\abs{X_{n,m}} > \epsilon} \to 0$ as $n \to \infty$. 
\end{enumerate}
\end{Theorem}

\begin{Remark}
Let $Y_1,Y_2,\ldots$ be i.i.d.\, $\E Y_1=0$, $\E Y_1^2=1$, and $X_{n,m}=Y_m/\sqrt{n}$. Then $\sum_{m=1}^n \E X_{n,m}^2=1$ and for all $\epsilon>0$, we have
\begin{align*}
\sum_{m=1}^n \E X_{n,m}^2 \indicator{\abs{X_{n,m}} > \epsilon}
= \E Y_1^2\indicator{|Y_1|>\epsilon\sqrt{n}} \xrightarrow{\text{DCT}} 0.
\end{align*}
Therefore, the CLT for i.i.d.\ sequence follows from Lindeberg's CLT.
\end{Remark}

\begin{proof}
By Chebyshev's inequality, we have
\begin{align*}
\P(\abs*{S_n} > M) \leq \ofrac{M^2} \sum_{m=1}^n \E X_{n,m}^2 \leq \frac{2}{M^2}, \ \forall n \ \text{sufficiently large}.
\end{align*}
Therefore, $(\P_{S_n})_{n\geq 1}$ is uniformly tight. From \cref{wk10:lem:uniformly_tight_and_varphi_conv_implies_convd}, to prove the theorem is equivalent to showing that the characteristic function $\varphi_{S_n}(t) \to e^{-\frac{t^2}{2}}$ as $n\to\infty$. We have
\begin{align*}
\log \varphi_{S_n}(t) 
= \log \left( \prod_{m=1}^n \E e^{\iu tX_{n,m}} \right) 
= \sum_{m=1}^n \log \left( 1 + \E e^{\iu tX_{n,m}} - 1 \right).
\end{align*}
By Taylor's series, we have the following elementary facts:
\begin{align} 
\abs*{\log(1+\xi) - \xi} 
\leq& \xi^2 \ \text{for}\ \abs*{\xi} \leq \frac{1}{2}, \label{Lindeberg_intmd_ineq1}\\ 
\abs*{e^{\iu a} - \sum_{k=0}^n \frac{(\iu a)^k}{k!}}
\leq& \frac{\abs*{a}^{n+1}}{(n+1)!} \text{ for } a\in\Real.\label{Lindeberg_intmd_ineq2}
\end{align}
From \cref{Lindeberg_intmd_ineq2} and $\E X_{n,m} = 0$, we have
\begin{align} \nonumber
&\abs*{\E e^{\iu t X_{n,m}} - 1} \\ \nonumber
=& \abs*{\E e^{\iu t X_{n,m}} - 1 - \iu t\E X_{n,m}} \\ \label{Lindeberg_intmd_ineq3}
\leq& \frac{t^2}{2} \E X_{n,m}^2 \\ \nonumber
\leq& \frac{t^2}{2} \epsilon^2 + \frac{t^2}{2}  \E X_{n,m}^2 \indicator{\abs*{X_{n,m}} > \epsilon}.
\end{align}
Because $\E X_{n,m}^2 \indicator{\abs*{X_{n,m} > \epsilon}} \to 0$ as $n \to \infty$, there exists $n_0(m)$ such that for $n \geq n_0(m)$, we have 
\begin{align*}
\abs*{\E e^{\iu t X_{n,m}} - 1}  \leq t^2\epsilon^2. 
\end{align*}
For $\epsilon \leq \dfrac{1}{t \sqrt{2}}$, 
\begin{align*}
\abs*{\E e^{\iu t X_{n,m}} - 1} \leq \frac{1}{2},
\end{align*}
thus from \cref{Lindeberg_intmd_ineq1,Lindeberg_intmd_ineq3}, we have
\begin{align*}
&\sum_{m=1}^n \abs*{\log \left(1+(\E e^{\iu t X_{n,m}} - 1)\right) - \left( \E e^{\iu t X_{n,m}} - 1 \right)} \\
\leq& \sum_{m=1}^n \abs*{\E e^{\iu t X_{n,m}} - 1 }^2  \\
\leq& \sum_{m=1}^n \frac{t^4}{4} \left( \E X_{n,m}^2 \right)^2 \\
\leq& \frac{t^4}{4} \max_{1 \leq m \leq n} \E X_{n,m}^2 \sum_{m=1}^n \E X_{n,m}^2.
\end{align*}
We have
\begin{align*}
\max_{1 \leq m \leq n} \E X_{n,m}^2 
\leq \epsilon^2 + \max_{1 \leq m \leq n} \E X_{n,m}^2 \indicator{\abs*{X_{n,m}} > \epsilon}
\to \epsilon^2 \ \text{as}\ n \to \infty,
\end{align*}
where the convergence follows from \cref{Lindeberg_condition2}. Combining the above result with \cref{Lindeberg_condition1}, we obtain
\begin{align}
\sum_{m=1}^n \abs*{\log \left(1+(\E e^{\iu t X_{n,m}} - 1)\right) - \left( \E e^{\iu t X_{n,m}} - 1 \right)}
\leq 
\frac{t^4}{2}\epsilon^2 \ \text{as}\ n \to \infty. \label{Lindeberg_intmd_ineq4}
\end{align}
Next we show
\begin{align*}
\abs*{\sum_{m=1}^n \left(\E e^{\iu t X_{n,m}} - 1 \right) + \frac{t^2}{2}} \to 0,
\end{align*}
which, due to \cref{Lindeberg_condition1}, is equivalent to showing that
\begin{align*}
\abs*{\sum_{m=1}^n \left(\E e^{\iu t X_{n,m}} - 1 \right) + \frac{t^2}{2} \sum_{m=1}^n \E X_{n,m}^2} \to 0.
\end{align*}
From the Taylor series expansion and using a similar argument in Lindeberg's method, we obtain
\begin{align*}
\abs*{ e^{\iu t X_{n,m}} - 1 - \iu t X_{n,m} + \frac{t^2}{2} X_{n,m}^2} \indicator{\abs*{X_{n,m}} > \epsilon}
\leq& t^2 X_{n,m}^2 \indicator{\abs*{X_{n,m}} > \epsilon}, \\
%
\abs*{ e^{\iu t X_{n,m}} - 1 - \iu t X_{n,m} + \frac{t^2}{2} X_{n,m}^2} \indicator{\abs*{X_{n,m}} \leq \epsilon}
\leq& \frac{t^3 \epsilon}{6} X_{n,m}^2.
\end{align*}
We then obtain
\begin{align*}
&\abs*{\sum_{m=1}^n \left( \E e^{\iu t X_{n,m}} - 1 \right) + \frac{t^2}{2} \sum_{m=1}^n \E X_{n,m}^2} \\
\leq& t^2 \sum_{m=1}^n \E X_{n,m}^2 \indicator{\abs*{X_{n,m}} > \epsilon} + \frac{t^3 \epsilon}{6} \sum_{m=1}^n \E X_{n,m}^2 \\
\leq& \frac{t^3 \epsilon}{6} \ \text{as}\ n \to \infty.
\end{align*}
Taking $\epsilon \to 0$, the theorem is proved.
\end{proof}
%
\begin{Theorem}[Berry-Essen Theorem]
Let $(X_m)_{m\geq1}$ be a sequence of independent random variables, $\E X_m = 0$. Let $\hat{F}_n$ be the cdf of $\dfrac{S_n}{\sqrt{\sum_{m=1}^n \E X_m^2}}$ and $G \sim \calN (0, 1)$. We have
\begin{align*}
\sup_x \abs*{\hat{F}_n(x) - G(x)} \leq \frac{10 \sum_{m=1}^n \E \abs{X_m}^3}{\left(\sum_{m=1}^n \E X_m^2\right)^{3/2}}.
\end{align*}
%
For the special case where $(X_m)$ is an i.i.d. sequence and $\E X_1^2 = \sigma^2$, we have
\begin{align*}
\sup_x \abs*{\hat{F}_n(x) - G(x)} \leq \frac{10 \E \abs{X_1}^3}{\sigma^3 \sqrt{n}}.
\end{align*}
\end{Theorem}
This theorem shows how fast the convergence is. The proof is omitted here as it is quite technical and tedious. Please see Durrett if interested. 

\section{Kolmogorov's Three Series Theorem}

\begin{Theorem}[Kolmogorov's Three Series Theorem] \label{Kolmogorov_Three_Series_Thm}
Let $(X_i)_{i\geq1}$ be a sequence of independent random variables and let $Z_i = X_i \indicator{\abs*{X_i} \leq 1}$. Then, $\sum_{i=1}^n X_i$ converges a.s.\ if and only if
\begin{enumerate}[(i)]
\item $\sum_{i \geq 1} \P(\abs{X_i} > 1) < \infty$. \label[condition]{Kolmogorov_condition_1}
\item $\sum_{i \geq 1} \E Z_i$ converges. \label[condition]{Kolmogorov_condition_2}
\item $\sum_{i \geq 1} \var Z_i < \infty$. \label[condition]{Kolmogorov_condition_3}
\end{enumerate} 
\end{Theorem}
%
\begin{proof}
``$\Leftarrow$'': Using \cref{Kolmogorov_condition_1}, we have
\begin{align*}
\sum_{i \geq 1} \P(X_i \neq Z_i) = \sum_{i \geq 1} \P(\abs{X_i} > 1) < \infty.
\end{align*}
From the Borel-Cantelli Lemma (\cref{wk6:lemma:borel_cantelli}), we have 
\begin{align*}
\P(X_i \neq Z_i\ \text{i.o.}) = 0.
\end{align*}
Therefore, 
\begin{align*}
\sum_{i \geq 1} X_i \ \text{converges}\
\iff
\sum_{i \geq 1} Z_i \ \text{converges}.
\end{align*}
Using \cref{Kolmogorov_condition_2}, we have
\begin{align*}
\sum_{i \geq 1} Z_i \ \text{converges}\ 
\iff
\sum_{i \geq 1} (Z_i - \E Z_i) \ \text{converges}.
\end{align*}
The proof now follows from \cref{Kolmogorov_condition_3} and the variance convergence criterion (\cref{wk7:thm:var_conv_criteria}).

``$\Rightarrow$'': Suppose that $\sum_{i \geq 1} X_i$ converges a.s.\\
We first show \cref{Kolmogorov_condition_1}. Since $\P(\abs*{X_i} > 1 \io) = 0$, the Borel-Cantelli Lemma (\cref{wk6:lemma:borel_cantelli}) yields 
\begin{align*}
\sum_{i \geq 1} \P(\abs*{X_i} > 1) < \infty. 
\end{align*}


We next show \cref{Kolmogorov_condition_3} by contradiction. From \cref{Kolmogorov_condition_1}, we have
\begin{align}\label{xconvz}
\sum_{i \geq 1}X_i \ \text{converges a.s.}\ \implies \sum_{i \geq 1}Z_i \ \text{converges a.s.}
\end{align}
Therefore, 
\begin{align*}
S(m,n) = \sum_{i=m}^n Z_i \to 0 \ \text{a.s as}\ m, n \to \infty. 
\end{align*}
Thus, for any $\delta > 0$, we have 
\begin{align} \label{Kolmogorov_intmd_ineq}
\P(\abs*{S(m,n)} > \delta) \leq \delta \ \text{for sufficiently large}\ m, n.
\end{align} 
Now assume $\sum_{i \geq 1} \var Z_i = \infty$. Then we have $\sigma^2(m,n) \triangleq \var(S(m,n)) = \sum_{i=m}^n \var(Z_i) \to \infty$ as $n \to \infty$ for fixed $m$. Let
\begin{align*}
T(m,n) 
= \frac{S(m,n) - \E S(m,n)}{\sigma(m,n)}
= \sum_{i=m}^n \frac{Z_i - \E Z_i}{\sigma(m,n)}.
\end{align*}
Let $\widetilde{Z}_i = Z_i - \E Z_i$. We have $\abs*{\widetilde{Z}_i} \leq 2$, and $\abs*{\frac{\widetilde{Z}_i}{\sigma(m,n)}} \to 0$ a.s as $n \to \infty$ because $\sigma(m,n) \to \infty$. For any $\epsilon>0$, $\exists n(m)$ such that for all $n \geq n(m)$, we have $\abs*{\frac{\widetilde{Z}_i}{\sigma(m,n)}} \leq \epsilon$ a.s. Therefore, we obtain
\begin{align*}
\sum_{i=m}^n \E[\frac{\widetilde{Z}_i^2}{\sigma^2(m,n)} \indicator{\abs*{\frac{\widetilde{Z}_i}{\sigma(m,n)}} > \epsilon}] = 0\ \as \text{ for}\ n \geq n(m).
\end{align*}
Furthermore, we have 
\begin{align*}
\sum_{m=1}^n \E[\frac{\widetilde{Z}_i^2}{\sigma^2(n,m)}] = 1.
\end{align*}
By Lindeberg's CLT, we therefore have
\begin{align}
T(m,n) \convd \calN (0,1) \ \text{as}\ m,n \to \infty.\label{TmnconvN}
\end{align}
But at the same time, we have
\begin{align*}
1-\delta \leq \P(\abs*{S(m,n)} \leq \delta) = \P(\abs*{T(m,n) + \frac{\E S(m,n)}{\sigma(m,n)}} \leq \frac{\delta}{\sigma(m,n)}).
\end{align*}
When $m, n \to \infty$, $\sigma(m,n) \to \infty$. Thus, $T(m,n)$ concentrates at a constant, which contradicts \cref{TmnconvN}. Therefore, the assumption that $\sum_{i \geq 1} \var Z_i = \infty$ is false.

Finally, we show \cref{Kolmogorov_condition_2}. From the variance convergence criterion (\cref{wk7:thm:var_conv_criteria}), $\sum_{i \geq 1}(Z_i - \E Z_i)$ converges a.s. Thus, convergence of $\sum_{i \geq 1} \E Z_i$ follows from \cref{xconvz}.
\end{proof}

\section{Levy's Equivalence Theorem}

\begin{Theorem}[Levy's Equivalence Theorem]\label{wk11:Levy's Equivalence Thm}
Let $(X_i)_{i\geq1}$ be a sequence of independent random variables. Then, $\sum_{i \geq 1} X_i$ converges a.s. $\iff$ converges in probability $\iff$ converges in distribution.
\end{Theorem}

To prove the \cref{wk11:Levy's Equivalence Thm}, we start with a few lemmas.


\begin{Lemma}\label{wk11:lem:Kolineq}
Suppose $X_1,X_2,\ldots$ are independent. Let $S_n=\sum_{i\leq n} X_i$. If $\P(|S_n-S_j|\geq a) \leq p < 1$ for all $j\leq n$, then for all $x > a$, we have
\begin{align*}
\P(\max_{1\leq j\leq n} |S_j| \geq x) \leq \ofrac{1-p} \P(|S_n| > x-a).
\end{align*}
\end{Lemma}
\begin{proof}
Let $\tau = \min\braces*{j\leq n : |S_j|\geq x}$ with $\tau=n+1$ if $|S_j| < x$ for all $j\leq n$. If $\tau=j$, then $|S_j|\geq x$ and 
\begin{align*}
\braces*{\omega : |S_n - S_j| < a,\ \tau=j} \subset \braces*{\omega : |S_n| >x-a,\ \tau=j}.
\end{align*}
This gives us
\begin{align*}
\P(\max_{j\leq n}|S_j| \geq x)
&=\P(\tau\leq n)\\
&=\sum_{j=1}^n \P(\tau=j)\\
&\leq \ofrac{1-p}\sum_{j=1}^n \P(|S_n-S_j|<a)\P(\tau=j)\\
&=\ofrac{1-p}\sum_{j=1}^n \P(|S_n-S_j|<a,\ \tau=j) \text{ since $\{\tau=j\}$ depends only on $X_1,\ldots,X_j$} \\
&\leq \ofrac{1-p}\sum_{j=1}^n \P(|S_n|>x-a,\ \tau=j)\\
&= \ofrac{1-p}\P(|S_n|>x-a).
\end{align*}
\end{proof}

\begin{Lemma}\label{wk11:lem:asiffmaxconvp}
$Y_n\to Y$ a.s. iff $\max_{i\geq n} |Y_i-Y| \convp 0$.
\end{Lemma}
\begin{proof}
We have $\max_{i\geq n} |Y_i-Y| \to 0\ \as\ \implies \max_{i\geq n} |Y_i-Y| \convp 0$.

To show the converse, let $M_n = \max_{i\geq n} |Y_i-Y|$, which is a decreasing sequence bounded below by 0. Therefore $M_n \downarrow M$ for some $M$, which implies that $\forall \epsilon>0$, $\P(M>\epsilon) \leq \P(M_n>\epsilon) \to 0$ as $n\to\infty$. Therefore, from continuity of $\P$, $\P(M=0)=1$ and $M_n\to 0$ a.s. This is equivalent to saying that $Y_n\to Y$ a.s.
\end{proof}

\begin{Lemma}\label{wk11:lem:Yconvpiff}
$(Y_n)$ converges in probability iff $\lim_{n,m\to\infty} \P(|Y_m-Y_n|\geq \epsilon) =0$ for all $\epsilon>0$.
\end{Lemma}
\begin{proof}
The proof in the ``$\Rightarrow$'' direction is trivial. To prove the converse, we note that the given condition implies that for all $k\geq1$, $\exists n(k)$ such that $\forall n,m \geq n(k)$, we have
\begin{align*}
\P(|Y_m-Y_n|\geq\ofrac{2^k}) \leq \ofrac{2^k}. 
\end{align*}
We can choose $n(k+1) \geq n(k)$ so that
\begin{align*}
\P(|Y_{n(k+1)}-Y_{n(k)}|\geq\ofrac{2^k}) \leq \ofrac{2^k}. 
\end{align*}
Summing over $k\geq 1$, we have
\begin{align*}
\sum_{k\geq1} \P(|Y_{n(k+1)}-Y_{n(k)}|\geq\ofrac{2^k}) \leq 1 <\infty. 
\end{align*}
The Borel-Cantelli Lemma (\cref{wk6:lemma:borel_cantelli}) implies that $\P(|Y_{n(k+1)}-Y_{n(k)}|\geq\ofrac{2^k} \io)=0$ and therefore $\exists l$ such that for all $k\geq l$, $|Y_{n(k+1)}-Y_{n(k)}|<\ofrac{2^k}$ a.s. and for all $j\geq k$, we have
\begin{align*}
|Y_{n(j)}-Y_{n(k)}|< \sum_{i\geq k} \ofrac{2^i} \to 0 \text{ as $k\to\infty$}.
\end{align*}
Therefore $(Y_{n(k)})_{k\geq1}$ is Cauchy a.s., and it converges since $\Real$ is complete. Hence, $\exists Y = \lim_{k\to\infty} Y_{n(k)}$. For any $\epsilon>0$, we can then choose $n(k)$ sufficiently large so that $\forall n\geq n(k)$,
\begin{align*}
\P(|Y_n-Y|\geq 2\epsilon)
&\leq \P(|Y_n - Y_{n(k)}|\geq\epsilon) + \P(|Y_{n(k)}-Y|\geq\epsilon) \\
&\leq 2\epsilon.
\end{align*}
The lemma is now proved.
\end{proof}

\begin{Lemma}\label{wk11:lem:XYut}
Suppose that $(\P_{X_n})$ and $(\P_{Y_n})$ are both uniformly tight. Then $(\P_{X_n+Y_n})$ is uniformly tight.
\end{Lemma}
\begin{proof}
Exercise.
\end{proof}

\begin{Lemma}\label{wk11:lem:Q0}
If $\P*\bbQ=\P$, then $\bbQ(\{0\})=1$.
\end{Lemma}
\begin{proof}
Let $X\sim\P$ and $Y\sim\bbQ$ be independent with characteristic functions $\varphi_X$ and $\varphi_Y$ respectively. Since $\P*\bbQ=\P$, we have $\varphi_X(t)\varphi_Y(t)=\varphi_X(t)$ for all $t\in\Real$. Since $\varphi_X(t)$ is continuous and $\varphi_X(0)=1$, $\exists \epsilon>0$ such that $|\varphi_X(t)|>0$ $\forall |t| \leq \epsilon$. This implies that $\varphi_Y(t)=1$ for such values of $t$. Therefore $\E[\cos tY]=1$ but since $\cos(\cdot) \leq 1$, we must have $tY = 0 \mod 2\pi$ a.s. 

Take $|s|, |t|\leq \epsilon$ with $s/t$ being irrational. Then for each $\omega$, we have
\begin{align*}
& tY(\omega) = 2\pi k,\ k\in\bbZ,\\
& sY(\omega) = 2\pi m,\ m\in\bbZ.
\end{align*}
If $Y(\omega) \ne 0$, $s/t = m/k$, a contradiction. Therefore $Y=0$ a.s.
\end{proof}

\begin{proof}[Proof of \cref{wk11:Levy's Equivalence Thm}]
We first show that $S_n=\sum_{i=1}^n X_i$ converges in probability implies convergence a.s. Suppose $S_n \convp S$, i.e., for all $\epsilon >0$, $\exists\ n(\epsilon)$ such that $\forall n\geq n(\epsilon)$,
\begin{align}\label{Snconvp}
\P(|S_n-S|>\epsilon) \leq \epsilon.
\end{align}
For $k,j\geq n(\epsilon)$, we have
\begin{align*}
\P(|S_k-S_j|\geq 2\epsilon) \leq \P(|S_k-S|\geq\epsilon) +\P(|S_j-S|\geq\epsilon) \leq 2\epsilon
\end{align*}
From \cref{wk11:lem:Kolineq}, we obtain
\begin{align*}
\P(\max_{n\leq j \leq k} |S_j-S_n| \geq 4\epsilon)
&\leq \ofrac{1-2\epsilon} \P(|S_k - S_n|\geq 2\epsilon)\\
&\leq \frac{2\epsilon}{1-2\epsilon}\\
&\leq 3\epsilon,
\end{align*}
for $\epsilon$ sufficiently small. The MCT (\cref{Monotone Convergence Theorem}) then yields
\begin{align*}
\P(\max_{j \geq n} |S_j-S_n| \geq 4\epsilon) \leq 3\epsilon.
\end{align*}
Together with \cref{Snconvp}, we finally have
\begin{align*}
\P(\max_{j \geq n} |S_j-S| \geq 5\epsilon) \leq 4\epsilon.
\end{align*} 
\cref{wk11:lem:asiffmaxconvp} then implies that $S_j \to S$ a.s.\ as $j\to\infty$.

We next show that $S_n$ converges in distribution implies convergence in probability. Suppose that $\P_{S_n} \convd \P$ for some $\P$. From \cref{wk9:lem:convd_implies_uniform_tightness}, $(\P_{S_n})_{n\geq1}$ is uniformly tight. Therefore from \cref{wk11:lem:XYut}, $(\P_{S_n-S_k})_{1\leq k\leq n}$ is uniformly tight. We proceed by contradiction. Suppose $\exists\ \epsilon>0$ and $(n(l))$, $(m(l))$ with $n(l)\leq m(l)$ $\forall l$ such that
\begin{align}\label{PYge}
\P(|S_{m(l)}-S_{n(l)}|>\epsilon) \geq \epsilon.
\end{align}
Let $Y_l=S_{m(l)}-S_{n(l)}$, then since $(\P_{Y_l})$ is uniformly tight, from Helly's Selection Theorem (\cref{wk9:thm:Helly}), $\exists\ (l(r))$ such that $\P_{Y_{l(r)}} \convd$ some distribution $\bbQ$. Since $S_{m(l(r))}=S_{n(l(r))}+Y_{l(r)}$ and $S_{n(l)}$, $Y_{l(r)}$ are independent, we have
\begin{align*}
\P_{S_{m(l(r))}} = \P_{S_{n(l(r))}}*\P_{Y_{l(r)}}.
\end{align*}
Taking $r\to\infty$, we then have $\P = \P*\bbQ$ from \cref{wk10:ex:P*Q}. From \cref{wk11:lem:Q0}, $\bbQ(\{0\})=1$, which implies that $\P(|Y_{l(r)}|>\epsilon) <\epsilon$ for $r$ sufficiently large. This contradicts \cref{PYge} and the proof is complete.
\end{proof}

\section{Poisson Convergence}

Let $X_{n,m}$, $n,m\geq 1$ be independent Bernoulli r.v.s with $\P(X_{n,m}=1)=p_{n,m}$. If $p_{n,m} \to 0$ sufficiently fast as $n\to\infty$, then
\begin{align*}
\sum_{m=1}^n \var X_{n,m} = \sum_{m=1}^n p_{n,m}(1-p_{n,m}) \to 0.
\end{align*}
This violates Lindeberg's CLT \cref{Lindeberg_condition1}, therefore we cannot apply the CLT here. However, we can still obtain a convergence in distribution result.

\begin{Theorem}\label{wk11:thm:Poisson convergence}
Suppose 
\begin{enumerate}[(i)]
	\item $\sum_{m=1}^n p_{n,m} \to \lambda \in (0,\infty)$ as $n\to\infty$, and
	\item $\max_{1\leq m\leq n} p_{n,m} \to 0$ as $n\to\infty$,
\end{enumerate}
then $S_n = \sum_{m=1}^n X_{n,m} \convd \Po{\lambda}$, where $\Po{\lambda}(k)= \dfrac{\lambda^k}{k!} e^{-\lambda}$ for $k\in\bbZ_{\geq0}$ is the Poisson distribution.
\end{Theorem} 
As in the proof of Lindeberg's CLT, the proof of this result proceeds via Levy's continuity theorem (\cref{wk10:thm:Levy's Continuity Theorem}). We need a preliminary lemma.

\begin{Lemma}\label{wk11:lem:prodbound}
If the complex numbers $z_i, w_i$, $i=1,\ldots,n$, are such that $|z_i|, |w_i| \leq \theta$, then
\begin{align*}
\abs*{\prod_{i=1}^n z_i - \prod_{i=1}^n w_i} \leq \theta^{n-1} \sum_{i=1}^n |z_i-w_i|.
\end{align*}
\end{Lemma}
\begin{proof}
We prove by induction on $n$. The result obviously holds for $n=1$. Suppose it is true for $n-1$. Then,
\begin{align*}
& \abs*{\prod_{i=1}^n z_i - \prod_{i=1}^n w_i} \\
& \leq \abs*{\prod_{i=1}^{n-1} z_i \cdot z_n - \prod_{i=1}^{n-1} w_i \cdot z_n} + \abs*{\prod_{i=1}^{n-1} w_i \cdot z_n - \prod_{i=1}^{n-1} w_i \cdot w_n} \\
& \leq \theta \abs*{\prod_{i=1}^{n-1} z_i - \prod_{i=1}^{n-1} w_i} + \theta^{n-1} \abs*{z_n - w_n} \\
& \leq \theta^{n-1} \sum_{i=1}^{n-1} |z_i-w_i| + \theta^{n-1} \abs*{z_n - w_n} \\
& = \theta^{n-1} \sum_{i=1}^n |z_i-w_i|.
\end{align*}
\end{proof}

\begin{proof}[Proof of \cref{wk11:lem:prodbound}]
The characteristic function of $\Po{\lambda}$ is $\exp(\lambda(e^{\iu t}-1))$ while that for $S_n$ is 
\begin{align*}
\E e^{\iu t S_n} = \prod_{m=1}^n \parens*{1 + p_{n,m}\parens*{e^{\iu t} - 1}}.
\end{align*}
We have
\begin{align*}
\abs*{\exp(p_{n,m}\parens*{e^{\iu t} - 1})} &= \exp(p_{n,m}\Re(e^{\iu t} - 1)) \\
&= \exp(p_{n,m}(\cos t - 1)) \leq 1,
\intertext{and also}
\abs*{1 + p_{n,m}(e^{\iu t} - 1)} &= \abs*{1 + p_{n,m}(\cos t - 1) + \iu\sin t} \leq 1,
\end{align*}
which together with \cref{wk11:lem:prodbound} yield
\begin{align*}
&\abs*{\prod_{m=1}^n \parens*{1 + p_{n,m}\parens*{e^{\iu t} - 1}} - \exp\parens*{\sum_{m=1}^n p_{n,m}(e^{\iu t}-1)}} \\
&\leq \sum_{m=1}^n \abs*{\exp\parens*{p_{n,m}\parens*{e^{\iu t} - 1}} - \parens*{1 + p_{n,m}\parens*{e^{\iu t} - 1}}} \\
&\leq \ofrac{2}\sum_{m=1}^n p_{n,m}^2 \abs*{e^{\iu t} - 1}^2 \\
&\leq 2 \max_{1\leq m\leq n} p_{n,m} \sum_{m=1}^n p_{n,m} \to 0 \text{ as $n\to\infty$}.
\end{align*}
Since $\exp\parens*{\sum_{m=1}^n p_{n,m}(e^{\iu t}-1)} \to \exp(\lambda(e^{\iu t}-1))$ as $n\to\infty$, the result follows from Levy's continuity theorem (\cref{wk10:thm:Levy's Continuity Theorem}).
\end{proof}

%\bibliography{mybib}
\bibliographystyle{alpha}

\end{document}
