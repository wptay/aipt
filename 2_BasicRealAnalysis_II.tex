\documentclass[12pt]{article}


\topmargin 0pt
\advance \topmargin by -\headheight
\advance \topmargin by -\headsep
\textheight 8.9in
\oddsidemargin 0pt
\evensidemargin \oddsidemargin
\marginparwidth 0.5in
\textwidth 6.5in

\parindent 0in
\parskip 1.5ex
%\renewcommand{\baselinestretch}{1.25}

\usepackage{xr-hyper}
\def\renewtheorem{}
\usepackage{lmodern}

% Note: this has been tested using MiKTeX 2.9. If you are getting errors, update your packages.
% LTeX: enabled=false

%%% Packages %%%
%\usepackage{setspace} % Double spaces document. Footnotes,
% figures, and tables will still be single spaced, however.
%\doublespacing
%\singlespacing
%\onehalfspacing
% \setstretch{1.5} % set double spacing to 1.5 or anything else.

\usepackage[T1]{fontenc}
\usepackage[utf8]{inputenc}
\usepackage{amsmath,amssymb,amsfonts,mathrsfs,bm}% Typical maths resource packages
\usepackage{mathtools}
\usepackage{amsthm}
\usepackage{nicefrac}
\usepackage[shortlabels]{enumitem}
\usepackage{graphicx}
\usepackage{epstopdf}
\DeclareGraphicsExtensions{.eps,.png,.jpg,.pdf}

\usepackage{url}
\usepackage{colortbl}
\usepackage{booktabs}
\usepackage{multirow}
\usepackage[table,dvipsnames]{xcolor}
\usepackage[normalem]{ulem}
\usepackage{xparse}
\usepackage{calc}
\usepackage{etoolbox}

\makeatletter
\@ifpackageloaded{natbib}{
	\relax
}{
	\usepackage{cite}
}
\makeatother

%\usepackage{pstricks}
%\usepackage{psfrag}
%\usepackage{syntonly}
%\syntaxonly
%\usepackage[style=base]{caption}
%\captionsetup{
%format = plain,
%font = footnotesize,
%labelfont = sc
%}


\usepackage{array}
\newcolumntype{L}[1]{>{\raggedright\let\newline\\\arraybackslash\hspace{0pt}}m{#1}}
\newcolumntype{C}[1]{>{\centering\let\newline\\\arraybackslash\hspace{0pt}}m{#1}}
\newcolumntype{R}[1]{>{\raggedleft\let\newline\\\arraybackslash\hspace{0pt}}m{#1}}

\makeatletter
\let\MYcaption\@makecaption
\makeatother
\usepackage[font=footnotesize]{subcaption}
\makeatletter
\let\@makecaption\MYcaption
\makeatother

\usepackage{glossaries}
\makeatletter
% copy old \gls and \glspl
\let\oldgls\gls
\let\oldglspl\glspl

% define a non space skipping version of \@ifnextchar
\newcommand\fussy@ifnextchar[3]{%
	\let\reserved@d=#1%
	\def\reserved@a{#2}%
	\def\reserved@b{#3}%
	\futurelet\@let@token\fussy@ifnch}
\def\fussy@ifnch{%
	\ifx\@let@token\reserved@d
		\let\reserved@c\reserved@a
	\else
		\let\reserved@c\reserved@b
	\fi
	\reserved@c}

\renewcommand{\gls}[1]{%
\oldgls{#1}\fussy@ifnextchar.{\@checkperiod}{\@}}
\renewcommand{\glspl}[1]{%
\oldglspl{#1}\fussy@ifnextchar.{\@checkperiod}{\@}}

\newcommand{\@checkperiod}[1]{%
	\ifnum\sfcode`\.=\spacefactor\else#1\fi
}
\makeatother

\newacronym{wrt}{w.r.t.}{with respect to}
\newacronym{RHS}{R.H.S.}{right-hand side}
\newacronym{LHS}{L.H.S.}{left-hand side}
\newacronym{iid}{i.i.d.}{independent and identically distributed}
%\newacronym{MIMO}{MIMO}{mulitple-input multiple-output}
%\newacronym{AOA}{AOA}{angle-of-arrival}
%\newacronym{AOD}{AOD}{angle-of-departure}
%\newacronym{LOS}{LOS}{line-of-sight}
%\newacronym{NLOS}{NLOS}{non-line-of-sight}
%\newacronym{TOA}{TOA}{time-of-arrival}
%\newacronym{TDOA}{TDOA}{time-difference-of-arrival}
%\newacronym{RSS}{RSS}{received signal strength}
%\newacronym{GNSS}{GNSS}{Global Navigation Satellite System}
%\newacronym{GSP}{GSP}{graph signal processing}
%\newacronym{ML}{ML}{machine learning}


%put the float package before hyperref and algorithm package after hyperref for hyperref to work correctly with algorithm
\usepackage{float}

\ifx\notloadhyperref\undefined
	\ifx\loadbibentry\undefined
		\usepackage[hidelinks,hypertexnames=false]{hyperref}
	\else
		\usepackage{bibentry}
		\makeatletter\let\saved@bibitem\@bibitem\makeatother
		\usepackage[hidelinks,hypertexnames=false]{hyperref}
		\makeatletter\let\@bibitem\saved@bibitem\makeatother
	\fi
\else
	\ifx\loadbibentry\undefined
		\relax
	\else
		\usepackage{bibentry}
	\fi
\fi

\usepackage[capitalize]{cleveref}
\crefname{equation}{}{}
\Crefname{equation}{}{}
\crefname{claim}{claim}{claims}
\crefname{step}{step}{steps}
\crefname{line}{line}{lines}
\crefname{condition}{condition}{conditions}
\crefname{dmath}{}{}
\crefname{dseries}{}{}
\crefname{dgroup}{}{}

\crefname{Problem}{Problem}{Problems}
\crefformat{Problem}{Problem~(#2#1#3)}
\crefrangeformat{Problem}{Problems~(#3#1#4) to~(#5#2#6)}

\crefname{Theorem}{Theorem}{Theorems}
\crefname{Corollary}{Corollary}{Corollaries}
\crefname{Proposition}{Proposition}{Propositions}
\crefname{Lemma}{Lemma}{Lemmas}
\crefname{Definition}{Definition}{Definitions}
\crefname{Example}{Example}{Examples}
\crefname{Assumption}{Assumption}{Assumptions}
\crefname{Remark}{Remark}{Remarks}
\crefname{Rem}{Remark}{Remarks}
\crefname{remarks}{Remarks}{Remarks}
\crefname{Appendix}{Appendix}{Appendices}
\crefname{Supplement}{Supplement}{Supplements}
\crefname{Exercise}{Exercise}{Exercises}
\crefname{Theorem_A}{Theorem}{Theorems}
\crefname{Corollary_A}{Corollary}{Corollaries}
\crefname{Proposition_A}{Proposition}{Propositions}
\crefname{Lemma_A}{Lemma}{Lemmas}
\crefname{Definition_A}{Definition}{Definitions}

\usepackage{crossreftools}
\ifx\notloadhyperref\undefined
	\pdfstringdefDisableCommands{%
		\let\Cref\crtCref
		\let\cref\crtcref
	}
\else
	\relax
\fi

\usepackage{algorithm,algorithmic}
\renewcommand{\algorithmicrequire}{\textbf{Input:}}
\renewcommand{\algorithmicensure}{\textbf{Output:}}

%may cause conflict with some packages like tikz, include manually if desired
%load after hyperref
\ifx\loadbreqn\undefined
	\relax
\else
	\usepackage{breqn}
\fi



%%%%%%%%%%%%%%%%%%%%%%%%%%%%%%%%%%%%%%%%%%%%%%%%


\interdisplaylinepenalty=2500   % To restore IEEEtran ability to automatically break
% within multiline equations, when using amsmath.

%%%%%%%%%%%%%%%%%%%%%%%%%%%%%%%%%%%%%%%%

%Theorem declarations

\ifx\renewtheorem\undefined
	% for use in main body
	\ifx\useTheoremCounter\undefined
		\newtheorem{Theorem}{Theorem}
		\newtheorem{Corollary}{Corollary}
		\newtheorem{Proposition}{Proposition}
		\newtheorem{Lemma}{Lemma}
	\else
		\newtheorem{Theorem}{Theorem}
		\newtheorem{Corollary}[Theorem]{Corollary}
		\newtheorem{Proposition}[Theorem]{Proposition}
	\fi

	\newtheorem{Definition}{Definition}
	\newtheorem{Example}{Example}
	\newtheorem{Remark}{Remark}
	\newtheorem{Assumption}{Assumption}
	\newtheorem{Exercise}{Exercise}

	% for use in the appendix
	\newtheorem{Theorem_A}{Theorem}[section]
	\newtheorem{Corollary_A}{Corollary}[section]
	\newtheorem{Proposition_A}{Proposition}[section]
	\newtheorem{Lemma_A}{Lemma}[section]
	\newtheorem{Definition_A}{Definition}[section]
	\newtheorem{Example_A}{Example}[section]
	\newtheorem{Remark_A}{Remark}[section]
	\newtheorem{Assumption_A}{Assumption}[section]
	\newtheorem{Exercise_A}{Exercise}[section]
\fi

% Remarks
\theoremstyle{remark}
\newtheorem{Rem}{Remark}
\theoremstyle{plain}

\newenvironment{remarks}{
	\begin{list}{\textit{Remark} \arabic{Rem}:~}{
			\setcounter{enumi}{\value{Rem}}
			\usecounter{Rem}
			\setcounter{Rem}{\value{enumi}}
			\setlength\labelwidth{0in}
			\setlength\labelsep{0in}
			\setlength\leftmargin{0in}
			\setlength\listparindent{0in}
			\setlength\itemindent{15pt}
		}
		}{
	\end{list}
}


% Special Headings
%\newtheorem*{Prop1}{Proposition 1} %needs amsthm

%\newtheoremstyle{nonum}{}{}{\itshape}{}{\bfseries}{.}{ }{#1 (\mdseries #3)}
%\theoremstyle{nonum}
%\newtheorem{Example**}{Example 1}

\newcommand{\EndExample}{{$\square$}}
%\renewcommand{\QED}{\QEDopen} % changes end of proof box to open box.

\newcommand{\qednew}{\nobreak \ifvmode \relax \else
		\ifdim\lastskip<1.5em \hskip-\lastskip
			\hskip1.5em plus0em minus0.5em \fi \nobreak
		\vrule height0.75em width0.5em depth0.25em\fi}


% achieves the functionality of \tag for subequations environment
\makeatletter
\newenvironment{varsubequations}[1]
{%
	\addtocounter{equation}{-1}%
	\begin{subequations}
		\renewcommand{\theparentequation}{#1}%
		\def\@currentlabel{#1}%
		}
		{%
	\end{subequations}\ignorespacesafterend
}
\makeatother


\newcommand{\ml}[1]{\begin{multlined}#1\end{multlined}}
\newcommand{\nn}{\nonumber\\ }


% Move down subscripts for some symbols like \chi
\NewDocumentCommand{\movedownsub}{e{^_}}{%
	\IfNoValueTF{#1}{%
		\IfNoValueF{#2}{^{}}% neither ^ nor _, do nothing; if no ^ but _, add ^{}
	}{%
		^{#1}% add superscript if present
	}%
	\IfNoValueF{#2}{_{#2}}% add subscript if present
}

% chi
\let\latexchi\chi
\RenewDocumentCommand{\chi}{}{\latexchi\movedownsub}


%Number sets
\newcommand{\Real}{\mathbb{R}}
\newcommand{\Nat}{\mathbb{N}}
\newcommand{\Rat}{\mathbb{Q}}
\newcommand{\Complex}{\mathbb{C}}

% imaginary number i
\newcommand{\iu}{\mathfrak{i}\mkern1mu}


% Calligraphic stuff
\newcommand{\calA}{\mathcal{A}}
\newcommand{\calB}{\mathcal{B}}
\newcommand{\calC}{\mathcal{C}}
\newcommand{\calD}{\mathcal{D}}
\newcommand{\calE}{\mathcal{E}}
\newcommand{\calF}{\mathcal{F}}
\newcommand{\calG}{\mathcal{G}}
\newcommand{\calH}{\mathcal{H}}
\newcommand{\calI}{\mathcal{I}}
\newcommand{\calJ}{\mathcal{J}}
\newcommand{\calK}{\mathcal{K}}
\newcommand{\calL}{\mathcal{L}}
\newcommand{\calM}{\mathcal{M}}
\newcommand{\calN}{\mathcal{N}}
\newcommand{\calO}{\mathcal{O}}
\newcommand{\calP}{\mathcal{P}}
\newcommand{\calQ}{\mathcal{Q}}
\newcommand{\calR}{\mathcal{R}}
\newcommand{\calS}{\mathcal{S}}
\newcommand{\calT}{\mathcal{T}}
\newcommand{\calU}{\mathcal{U}}
\newcommand{\calV}{\mathcal{V}}
\newcommand{\calW}{\mathcal{W}}
\newcommand{\calX}{\mathcal{X}}
\newcommand{\calY}{\mathcal{Y}}
\newcommand{\calZ}{\mathcal{Z}}

% Boldface stuff
\newcommand{\ba}{\mathbf{a}}
\newcommand{\bA}{\mathbf{A}}
\newcommand{\bb}{\mathbf{b}}
\newcommand{\bB}{\mathbf{B}}
\newcommand{\bc}{\mathbf{c}}
\newcommand{\bC}{\mathbf{C}}
\newcommand{\bd}{\mathbf{d}}
\newcommand{\bD}{\mathbf{D}}
\newcommand{\be}{\mathbf{e}}
\newcommand{\bE}{\mathbf{E}}
\newcommand{\boldf}{\mathbf{f}}
\newcommand{\bF}{\mathbf{F}}
\newcommand{\bg}{\mathbf{g}}
\newcommand{\bG}{\mathbf{G}}
\newcommand{\bh}{\mathbf{h}}
\newcommand{\bH}{\mathbf{H}}
\newcommand{\bi}{\mathbf{i}}
\newcommand{\bI}{\mathbf{I}}
\newcommand{\bj}{\mathbf{j}}
\newcommand{\bJ}{\mathbf{J}}
\newcommand{\bk}{\mathbf{k}}
\newcommand{\bK}{\mathbf{K}}
\newcommand{\bl}{\mathbf{l}}
\newcommand{\bL}{\mathbf{L}}
\newcommand{\boldm}{\mathbf{m}}
\newcommand{\bM}{\mathbf{M}}
\newcommand{\bn}{\mathbf{n}}
\newcommand{\bN}{\mathbf{N}}
\newcommand{\bo}{\mathbf{o}}
\newcommand{\bO}{\mathbf{O}}
\newcommand{\bp}{\mathbf{p}}
\newcommand{\bP}{\mathbf{P}}
\newcommand{\bq}{\mathbf{q}}
\newcommand{\bQ}{\mathbf{Q}}
\newcommand{\br}{\mathbf{r}}
\newcommand{\bR}{\mathbf{R}}
\newcommand{\bs}{\mathbf{s}}
\newcommand{\bS}{\mathbf{S}}
\newcommand{\bt}{\mathbf{t}}
\newcommand{\bT}{\mathbf{T}}
\newcommand{\bu}{\mathbf{u}}
\newcommand{\bU}{\mathbf{U}}
\newcommand{\bv}{\mathbf{v}}
\newcommand{\bV}{\mathbf{V}}
\newcommand{\bw}{\mathbf{w}}
\newcommand{\bW}{\mathbf{W}}
\newcommand{\bx}{\mathbf{x}}
\newcommand{\bX}{\mathbf{X}}
\newcommand{\by}{\mathbf{y}}
\newcommand{\bY}{\mathbf{Y}}
\newcommand{\bz}{\mathbf{z}}
\newcommand{\bZ}{\mathbf{Z}}


\newcommand{\mba}{\bm{a}}
\newcommand{\mbA}{\bm{A}}
\newcommand{\mbb}{\bm{b}}
\newcommand{\mbB}{\bm{B}}
\newcommand{\mbc}{\bm{c}}
\newcommand{\mbC}{\bm{C}}
\newcommand{\mbd}{\bm{d}}
\newcommand{\mbD}{\bm{D}}
\newcommand{\mbe}{\bm{e}}
\newcommand{\mbE}{\bm{E}}
\newcommand{\mbf}{\bm{f}}
\newcommand{\mbF}{\bm{F}}
\newcommand{\mbg}{\bm{g}}
\newcommand{\mbG}{\bm{G}}
\newcommand{\mbh}{\bm{h}}
\newcommand{\mbH}{\bm{H}}
\newcommand{\mbi}{\bm{i}}
\newcommand{\mbI}{\bm{I}}
\newcommand{\mbj}{\bm{j}}
\newcommand{\mbJ}{\bm{J}}
\newcommand{\mbk}{\bm{k}}
\newcommand{\mbK}{\bm{K}}
\newcommand{\mbl}{\bm{l}}
\newcommand{\mbL}{\bm{L}}
\newcommand{\mbm}{\bm{m}}
\newcommand{\mbM}{\bm{M}}
\newcommand{\mbn}{\bm{n}}
\newcommand{\mbN}{\bm{N}}
\newcommand{\mbo}{\bm{o}}
\newcommand{\mbO}{\bm{O}}
\newcommand{\mbp}{\bm{p}}
\newcommand{\mbP}{\bm{P}}
\newcommand{\mbq}{\bm{q}}
\newcommand{\mbQ}{\bm{Q}}
\newcommand{\mbr}{\bm{r}}
\newcommand{\mbR}{\bm{R}}
\newcommand{\mbs}{\bm{s}}
\newcommand{\mbS}{\bm{S}}
\newcommand{\mbt}{\bm{t}}
\newcommand{\mbT}{\bm{T}}
\newcommand{\mbu}{\bm{u}}
\newcommand{\mbU}{\bm{U}}
\newcommand{\mbv}{\bm{v}}
\newcommand{\mbV}{\bm{V}}
\newcommand{\mbw}{\bm{w}}
\newcommand{\mbW}{\bm{W}}
\newcommand{\mbx}{\bm{x}}
\newcommand{\mbX}{\bm{X}}
\newcommand{\mby}{\bm{y}}
\newcommand{\mbY}{\bm{Y}}
\newcommand{\mbz}{\bm{z}}
\newcommand{\mbZ}{\bm{Z}}

% Numbers bb font
\newcommand{\bbA}{\mathbb{A}}
\newcommand{\bbB}{\mathbb{B}}
\newcommand{\bbC}{\mathbb{C}}
\newcommand{\bbD}{\mathbb{D}}
\newcommand{\bbE}{\mathbb{E}}
\newcommand{\bbF}{\mathbb{F}}
\newcommand{\bbG}{\mathbb{G}}
\newcommand{\bbH}{\mathbb{H}}
\newcommand{\bbI}{\mathbb{I}}
\newcommand{\bbJ}{\mathbb{J}}
\newcommand{\bbK}{\mathbb{K}}
\newcommand{\bbL}{\mathbb{L}}
\newcommand{\bbM}{\mathbb{M}}
\newcommand{\bbN}{\mathbb{N}}
\newcommand{\bbO}{\mathbb{O}}
\newcommand{\bbP}{\mathbb{P}}
\newcommand{\bbQ}{\mathbb{Q}}
\newcommand{\bbR}{\mathbb{R}}
\newcommand{\bbS}{\mathbb{S}}
\newcommand{\bbT}{\mathbb{T}}
\newcommand{\bbU}{\mathbb{U}}
\newcommand{\bbV}{\mathbb{V}}
\newcommand{\bbW}{\mathbb{W}}
\newcommand{\bbX}{\mathbb{X}}
\newcommand{\bbY}{\mathbb{Y}}
\newcommand{\bbZ}{\mathbb{Z}}

% Mathfrak font
\newcommand{\frakA}{\mathfrak{A}}
\newcommand{\frakB}{\mathfrak{B}}
\newcommand{\frakC}{\mathfrak{C}}
\newcommand{\frakD}{\mathfrak{D}}
\newcommand{\frakE}{\mathfrak{E}}
\newcommand{\frakF}{\mathfrak{F}}
\newcommand{\frakG}{\mathfrak{G}}
\newcommand{\frakH}{\mathfrak{H}}
\newcommand{\frakI}{\mathfrak{I}}
\newcommand{\frakJ}{\mathfrak{J}}
\newcommand{\frakK}{\mathfrak{K}}
\newcommand{\frakL}{\mathfrak{L}}
\newcommand{\frakM}{\mathfrak{M}}
\newcommand{\frakN}{\mathfrak{N}}
\newcommand{\frakO}{\mathfrak{O}}
\newcommand{\frakP}{\mathfrak{P}}
\newcommand{\frakQ}{\mathfrak{Q}}
\newcommand{\frakR}{\mathfrak{R}}
\newcommand{\frakS}{\mathfrak{S}}
\newcommand{\frakT}{\mathfrak{T}}
\newcommand{\frakU}{\mathfrak{U}}
\newcommand{\frakV}{\mathfrak{V}}
\newcommand{\frakW}{\mathfrak{W}}
\newcommand{\frakX}{\mathfrak{X}}
\newcommand{\frakY}{\mathfrak{Y}}
\newcommand{\frakZ}{\mathfrak{Z}}

% Mathscr
\newcommand{\scA}{\mathscr{A}}
\newcommand{\scB}{\mathscr{B}}
\newcommand{\scC}{\mathscr{C}}
\newcommand{\scD}{\mathscr{D}}
\newcommand{\scE}{\mathscr{E}}
\newcommand{\scF}{\mathscr{F}}
\newcommand{\scG}{\mathscr{G}}
\newcommand{\scH}{\mathscr{H}}
\newcommand{\scI}{\mathscr{I}}
\newcommand{\scJ}{\mathscr{J}}
\newcommand{\scK}{\mathscr{K}}
\newcommand{\scL}{\mathscr{L}}
\newcommand{\scM}{\mathscr{M}}
\newcommand{\scN}{\mathscr{N}}
\newcommand{\scO}{\mathscr{O}}
\newcommand{\scP}{\mathscr{P}}
\newcommand{\scQ}{\mathscr{Q}}
\newcommand{\scR}{\mathscr{R}}
\newcommand{\scS}{\mathscr{S}}
\newcommand{\scT}{\mathscr{T}}
\newcommand{\scU}{\mathscr{U}}
\newcommand{\scV}{\mathscr{V}}
\newcommand{\scW}{\mathscr{W}}
\newcommand{\scX}{\mathscr{X}}
\newcommand{\scY}{\mathscr{Y}}
\newcommand{\scZ}{\mathscr{Z}}


% define some useful uppercase Greek letters in regular and bold sf
\DeclareSymbolFont{bsfletters}{OT1}{cmss}{bx}{n}
\DeclareSymbolFont{ssfletters}{OT1}{cmss}{m}{n}
\DeclareMathSymbol{\bsfGamma}{0}{bsfletters}{'000}
\DeclareMathSymbol{\ssfGamma}{0}{ssfletters}{'000}
\DeclareMathSymbol{\bsfDelta}{0}{bsfletters}{'001}
\DeclareMathSymbol{\ssfDelta}{0}{ssfletters}{'001}
\DeclareMathSymbol{\bsfTheta}{0}{bsfletters}{'002}
\DeclareMathSymbol{\ssfTheta}{0}{ssfletters}{'002}
\DeclareMathSymbol{\bsfLambda}{0}{bsfletters}{'003}
\DeclareMathSymbol{\ssfLambda}{0}{ssfletters}{'003}
\DeclareMathSymbol{\bsfXi}{0}{bsfletters}{'004}
\DeclareMathSymbol{\ssfXi}{0}{ssfletters}{'004}
\DeclareMathSymbol{\bsfPi}{0}{bsfletters}{'005}
\DeclareMathSymbol{\ssfPi}{0}{ssfletters}{'005}
\DeclareMathSymbol{\bsfSigma}{0}{bsfletters}{'006}
\DeclareMathSymbol{\ssfSigma}{0}{ssfletters}{'006}
\DeclareMathSymbol{\bsfUpsilon}{0}{bsfletters}{'007}
\DeclareMathSymbol{\ssfUpsilon}{0}{ssfletters}{'007}
\DeclareMathSymbol{\bsfPhi}{0}{bsfletters}{'010}
\DeclareMathSymbol{\ssfPhi}{0}{ssfletters}{'010}
\DeclareMathSymbol{\bsfPsi}{0}{bsfletters}{'011}
\DeclareMathSymbol{\ssfPsi}{0}{ssfletters}{'011}
\DeclareMathSymbol{\bsfOmega}{0}{bsfletters}{'012}
\DeclareMathSymbol{\ssfOmega}{0}{ssfletters}{'012}


% Greek
\newcommand{\balpha}{\bm{\alpha}}
\newcommand{\bbeta}{\bm{\beta}}
\newcommand{\bgamma}{\bm{\gamma}}
\newcommand{\bdelta}{\bm{\delta}}
\newcommand{\btheta}{\bm{\theta}}
\newcommand{\bmu}{\bm{\mu}}
\newcommand{\bnu}{\bm{\nu}}
\newcommand{\btau}{\bm{\tau}}
\newcommand{\bpi}{\bm{\pi}}
\newcommand{\bepsilon}{\bm{\epsilon}}
\newcommand{\bvarepsilon}{\bm{\varepsilon}}
\newcommand{\bsigma}{\bm{\sigma}}
\newcommand{\bvarsigma}{\bm{\varsigma}}
\newcommand{\bzeta}{\bm{\zeta}}
\newcommand{\bmeta}{\bm{\eta}}
\newcommand{\bkappa}{\bm{\kappa}}
\newcommand{\bchi}{\bm{\latexchi}\movedownsub}
\newcommand{\bphi}{\bm{\phi}}
\newcommand{\bpsi}{\bm{\psi}}
\newcommand{\bomega}{\bm{\omega}}
\newcommand{\bxi}{\bm{\xi}}
\newcommand{\blambda}{\bm{\lambda}}
\newcommand{\brho}{\bm{\rho}}

\newcommand{\bGamma}{\bm{\Gamma}}
\newcommand{\bLambda}{\bm{\Lambda}}
\newcommand{\bSigma	}{\bm{\Sigma}}
\newcommand{\bPsi}{\bm{\Psi}}
\newcommand{\bDelta}{\bm{\Delta}}
\newcommand{\bXi}{\bm{\Xi}}
\newcommand{\bUpsilon}{\bm{\Upsilon}}
\newcommand{\bOmega}{\bm{\Omega}}
\newcommand{\bPhi}{\bm{\Phi}}
\newcommand{\bPi}{\bm{\Pi}}
\newcommand{\bTheta}{\bm{\Theta}}

\newcommand{\talpha}{\widetilde{\alpha}}
\newcommand{\tbeta}{\widetilde{\beta}}
\newcommand{\tgamma}{\widetilde{\gamma}}
\newcommand{\tdelta}{\widetilde{\delta}}
\newcommand{\ttheta}{\widetilde{\theta}}
\newcommand{\tmu}{\widetilde{\mu}}
\newcommand{\tnu}{\widetilde{\nu}}
\newcommand{\ttau}{\widetilde{\tau}}
\newcommand{\tpi}{\widetilde{\pi}}
\newcommand{\tepsilon}{\widetilde{\epsilon}}
\newcommand{\tvarepsilon}{\widetilde{\varepsilon}}
\newcommand{\tsigma}{\widetilde{\sigma}}
\newcommand{\tvarsigma}{\widetilde{\varsigma}}
\newcommand{\tzeta}{\widetilde{\zeta}}
\newcommand{\tmeta}{\widetilde{\eta}}
\newcommand{\tkappa}{\widetilde{\kappa}}
\newcommand{\tchi}{\widetilde{\latexchi}\movedownsub}
\newcommand{\tphi}{\widetilde{\phi}}
\newcommand{\tpsi}{\widetilde{\psi}}
\newcommand{\tomega}{\widetilde{\omega}}
\newcommand{\txi}{\widetilde{\xi}}
\newcommand{\tlambda}{\widetilde{\lambda}}
\newcommand{\trho}{\widetilde{\rho}}

\newcommand{\tGamma}{\widetilde{\Gamma}}
\newcommand{\tDelta}{\widetilde{\Delta}}
\newcommand{\tTheta}{\widetilde{\Theta}}
\newcommand{\tPi}{\widetilde{\Pi}}
\newcommand{\tSigma}{\widetilde{\Sigma}}
\newcommand{\tPhi}{\widetilde{\Phi}}
\newcommand{\tPsi}{\widetilde{\Psi}}
\newcommand{\tOmega}{\widetilde{\Omega}}
\newcommand{\tXi}{\widetilde{\Xi}}
\newcommand{\tLambda}{\widetilde{\Lambda}}

\newcommand{\tbalpha}{\widetilde{\balpha}}
\newcommand{\tbbeta}{\widetilde{\bbeta}}
\newcommand{\tbgamma}{\widetilde{\bgamma}}
\newcommand{\tbdelta}{\widetilde{\bdelta}}
\newcommand{\tbtheta}{\widetilde{\btheta}}
\newcommand{\tbmu}{\widetilde{\bmu}}
\newcommand{\tbnu}{\widetilde{\bnu}}
\newcommand{\tbtau}{\widetilde{\btbau}}
\newcommand{\tbpi}{\widetilde{\bpi}}
\newcommand{\tbepsilon}{\widetilde{\bepsilon}}
\newcommand{\tbvarepsilon}{\widetilde{\bvarepsilon}}
\newcommand{\tbsigma}{\widetilde{\bsigma}}
\newcommand{\tbvarsigma}{\widetilde{\bvarsigma}}
\newcommand{\tbzeta}{\widetilde{\bzeta}}
\newcommand{\tbmeta}{\widetilde{\beta}}
\newcommand{\tbkappa}{\widetilde{\bkappa}}
\newcommand{\tbchi}{\widetilde\bm{\latexchi}\movedownsub}
\newcommand{\tbphi}{\widetilde{\bphi}}
\newcommand{\tbpsi}{\widetilde{\bpsi}}
\newcommand{\tbomega}{\widetilde{\bomega}}
\newcommand{\tbxi}{\widetilde{\bxi}}
\newcommand{\tblambda}{\widetilde{\blambda}}
\newcommand{\tbrho}{\widetilde{\brho}}

\newcommand{\tbGamma}{\widetilde{\bGamma}}
\newcommand{\tbDelta}{\widetilde{\bDelta}}
\newcommand{\tbTheta}{\widetilde{\bTheta}}
\newcommand{\tbPi}{\widetilde{\bPi}}
\newcommand{\tbSigma}{\widetilde{\bSigma}}
\newcommand{\tbPhi}{\widetilde{\bPhi}}
\newcommand{\tbPsi}{\widetilde{\bPsi}}
\newcommand{\tbOmega}{\widetilde{\bOmega}}
\newcommand{\tbXi}{\widetilde{\bXi}}
\newcommand{\tbLambda}{\widetilde{\bLambda}}

\newcommand{\halpha}{\widehat{\alpha}}
\newcommand{\hbeta}{\widehat{\beta}}
\newcommand{\hgamma}{\widehat{\gamma}}
\newcommand{\hdelta}{\widehat{\delta}}
\newcommand{\htheta}{\widehat{\theta}}
\newcommand{\hmu}{\widehat{\mu}}
\newcommand{\hnu}{\widehat{\nu}}
\newcommand{\htau}{\widehat{\tau}}
\newcommand{\hpi}{\widehat{\pi}}
\newcommand{\hepsilon}{\widehat{\epsilon}}
\newcommand{\hvarepsilon}{\widehat{\varepsilon}}
\newcommand{\hsigma}{\widehat{\sigma}}
\newcommand{\hvarsigma}{\widehat{\varsigma}}
\newcommand{\hzeta}{\widehat{\zeta}}
\newcommand{\hmeta}{\widehat{\eta}}
\newcommand{\hkappa}{\widehat{\kappa}}
\newcommand{\hchi}{\widehat{\latexchi}\movedownsub}
\newcommand{\hphi}{\widehat{\phi}}
\newcommand{\hpsi}{\widehat{\psi}}
\newcommand{\homega}{\widehat{\omega}}
\newcommand{\hxi}{\widehat{\xi}}
\newcommand{\hlambda}{\widehat{\lambda}}
\newcommand{\hrho}{\widehat{\rho}}

\newcommand{\hGamma}{\widehat{\Gamma}}
\newcommand{\hDelta}{\widehat{\Delta}}
\newcommand{\hTheta}{\widehat{\Theta}}
\newcommand{\hPi}{\widehat{\Pi}}
\newcommand{\hSigma}{\widehat{\Sigma}}
\newcommand{\hPhi}{\widehat{\Phi}}
\newcommand{\hPsi}{\widehat{\Psi}}
\newcommand{\hOmega}{\widehat{\Omega}}
\newcommand{\hXi}{\widehat{\Xi}}
\newcommand{\hLambda}{\widehat{\Lambda}}

\newcommand{\hbalpha}{\widehat{\balpha}}
\newcommand{\hbbeta}{\widehat{\bbeta}}
\newcommand{\hbgamma}{\widehat{\bgamma}}
\newcommand{\hbdelta}{\widehat{\bdelta}}
\newcommand{\hbtheta}{\widehat{\btheta}}
\newcommand{\hbmu}{\widehat{\bmu}}
\newcommand{\hbnu}{\widehat{\bnu}}
\newcommand{\hbtau}{\widehat{\btau}}
\newcommand{\hbpi}{\widehat{\bpi}}
\newcommand{\hbepsilon}{\widehat{\bepsilon}}
\newcommand{\hbvarepsilon}{\widehat{\bvarepsilon}}
\newcommand{\hbsigma}{\widehat{\bsigma}}
\newcommand{\hbvarsigma}{\widehat{\bvarsigma}}
\newcommand{\hbzeta}{\widehat{\bzeta}}
\newcommand{\hbmeta}{\widehat{\beta}}
\newcommand{\hbkappa}{\widehat{\bkappa}}
\newcommand{\hbchi}{\widehat\bm{\latexchi}\movedownsub}
\newcommand{\hbphi}{\widehat{\bphi}}
\newcommand{\hbpsi}{\widehat{\bpsi}}
\newcommand{\hbomega}{\widehat{\bomega}}
\newcommand{\hbxi}{\widehat{\bxi}}
\newcommand{\hblambda}{\widehat{\blambda}}
\newcommand{\hbrho}{\widehat{\brho}}

\newcommand{\hbGamma}{\widehat{\bGamma}}
\newcommand{\hbDelta}{\widehat{\bDelta}}
\newcommand{\hbTheta}{\widehat{\bTheta}}
\newcommand{\hbPi}{\widehat{\bPi}}
\newcommand{\hbSigma}{\widehat{\bSigma}}
\newcommand{\hbPhi}{\widehat{\bPhi}}
\newcommand{\hbPsi}{\widehat{\bPsi}}
\newcommand{\hbOmega}{\widehat{\bOmega}}
\newcommand{\hbXi}{\widehat{\bXi}}
\newcommand{\hbLambda}{\widehat{\bLambda}}

\makeatletter
\newcommand*\rel@kern[1]{\kern#1\dimexpr\macc@kerna}
\newcommand*\widebar[1]{%
  \begingroup
  \def\mathaccent##1##2{%
    \rel@kern{0.8}%
    \overline{\rel@kern{-0.8}\macc@nucleus\rel@kern{0.2}}%
    \rel@kern{-0.2}%
  }%
  \macc@depth\@ne
  \let\math@bgroup\@empty \let\math@egroup\macc@set@skewchar
  \mathsurround\z@ \frozen@everymath{\mathgroup\macc@group\relax}%
  \macc@set@skewchar\relax
  \let\mathaccentV\macc@nested@a
  \macc@nested@a\relax111{#1}%
  \endgroup
}
\makeatother

\newcommand{\barbalpha}{\widebar{\balpha}}
\newcommand{\barbbeta}{\widebar{\bbeta}}
\newcommand{\barbgamma}{\widebar{\bgamma}}
\newcommand{\barbdelta}{\widebar{\bdelta}}
\newcommand{\barbtheta}{\widebar{\btheta}}
\newcommand{\barbmu}{\widebar{\bmu}}
\newcommand{\barbnu}{\widebar{\bnu}}
\newcommand{\barbtau}{\widebar{\btau}}
\newcommand{\barbpi}{\widebar{\bpi}}
\newcommand{\barbepsilon}{\widebar{\bepsilon}}
\newcommand{\barbvarepsilon}{\widebar{\bvarepsilon}}
\newcommand{\barbsigma}{\widebar{\bsigma}}
\newcommand{\barbvarsigma}{\widebar{\bvarsigma}}
\newcommand{\barbzeta}{\widebar{\bzeta}}
\newcommand{\barbmeta}{\widebar{\beta}}
\newcommand{\barbkappa}{\widebar{\bkappa}}
\newcommand{\barbchi}{\bar\bm{\latexchi}\movedownsub}
\newcommand{\barbphi}{\widebar{\bphi}}
\newcommand{\barbpsi}{\widebar{\bpsi}}
\newcommand{\barbomega}{\widebar{\bomega}}
\newcommand{\barbxi}{\widebar{\bxi}}
\newcommand{\barblambda}{\widebar{\blambda}}
\newcommand{\barbrho}{\widebar{\brho}}

\newcommand{\barbGamma}{\widebar{\bGamma}}
\newcommand{\barbDelta}{\widebar{\bDelta}}
\newcommand{\barbTheta}{\widebar{\bTheta}}
\newcommand{\barbPi}{\widebar{\bPi}}
\newcommand{\barbSigma}{\widebar{\bSigma}}
\newcommand{\barbPhi}{\widebar{\bPhi}}
\newcommand{\barbPsi}{\widebar{\bPsi}}
\newcommand{\barbOmega}{\widebar{\bOmega}}
\newcommand{\barbXi}{\widebar{\bXi}}
\newcommand{\barbLambda}{\widebar{\bLambda}}

%MathOperator
\DeclareMathOperator*{\argmax}{arg\,max}
\DeclareMathOperator*{\argmin}{arg\,min}
\DeclareMathOperator*{\argsup}{arg\,sup}
\DeclareMathOperator*{\arginf}{arg\,inf}
\DeclareMathOperator*{\minimize}{minimize}
\DeclareMathOperator*{\maximize}{maximize}
\DeclareMathOperator{\st}{s.t.\ }
%\DeclareMathOperator{\st}{subject\,\,to}
\DeclareMathOperator{\as}{a.s.}
\DeclareMathOperator{\const}{const}
\DeclareMathOperator{\diag}{diag}
\DeclareMathOperator{\cum}{cum}
\DeclareMathOperator{\sgn}{sgn}
\DeclareMathOperator{\tr}{tr}
\DeclareMathOperator{\Tr}{Tr}
\DeclareMathOperator{\spn}{span}
\DeclareMathOperator{\supp}{supp}
\DeclareMathOperator{\adj}{adj}
\DeclareMathOperator{\var}{var}
\DeclareMathOperator{\Vol}{Vol}
\DeclareMathOperator{\cov}{cov}
\DeclareMathOperator{\corr}{corr}
\DeclareMathOperator{\sech}{sech}
\DeclareMathOperator{\sinc}{sinc}
\DeclareMathOperator{\rank}{rank}
\DeclareMathOperator{\poly}{poly}
\DeclareMathOperator{\vect}{vec}
\DeclareMathOperator{\conv}{conv}
\DeclareMathOperator*{\lms}{l.i.m.\,}
\DeclareMathOperator*{\esssup}{ess\,sup}
\DeclareMathOperator*{\essinf}{ess\,inf}
\DeclareMathOperator{\sign}{sign}
\DeclareMathOperator{\eig}{eig}
\DeclareMathOperator{\ima}{im}
\DeclareMathOperator{\Mod}{mod}

%Paired delimiters
\newcommand{\ifbcdot}[1]{\ifblank{#1}{\cdot}{#1}}

\DeclarePairedDelimiterX\abs[1]{\lvert}{\rvert}{\ifbcdot{#1}}
\DeclarePairedDelimiterX\parens[1]{(}{)}{\ifbcdot{#1}}
\DeclarePairedDelimiterX\brk[1]{[}{]}{\ifbcdot{#1}}
\DeclarePairedDelimiterX\braces[1]{\{}{\}}{\ifbcdot{#1}}
\DeclarePairedDelimiterX\angles[1]{\langle}{\rangle}{#1}
\DeclarePairedDelimiterX\ip[2]{\langle}{\rangle}{\ifbcdot{#1},\ifbcdot{#2}}
\DeclarePairedDelimiterX\norm[1]{\lVert}{\rVert}{\ifbcdot{#1}}
\DeclarePairedDelimiterX\ceil[1]{\lceil}{\rceil}{\ifbcdot{#1}}
\DeclarePairedDelimiterX\floor[1]{\lfloor}{\rfloor}{\ifbcdot{#1}}

\DeclarePairedDelimiterXPP\trace[1]{\operatorname{Tr}}{(}{)}{}{\ifbcdot{#1}} % column vector
\DeclarePairedDelimiterXPP\col[1]{\operatorname{col}}{\{}{\}}{}{\ifbcdot{#1}} % column vector
\DeclarePairedDelimiterXPP\row[1]{\operatorname{row}}{\{}{\}}{}{\ifbcdot{#1}} % row vector
\DeclarePairedDelimiterXPP\erf[1]{\operatorname{erf}}{(}{)}{}{\ifbcdot{#1}}
\DeclarePairedDelimiterXPP\erfc[1]{\operatorname{erfc}}{(}{)}{}{\ifbcdot{#1}}
\DeclarePairedDelimiterXPP\KLD[2]{D}{(}{)}{}{\ifbcdot{#1}\, \delimsize\|\, \ifbcdot{#2}} % KL divergence
\DeclarePairedDelimiterXPP\op[2]{\operatorname{#1}}{(}{)}{}{#2} % general operator

% Math relations
\newcommand{\convp}{\stackrel{\mathrm{p}}{\longrightarrow}}
\newcommand{\convas}{\stackrel{\mathrm{a.s.}}{\longrightarrow}}
\newcommand{\convd}{\stackrel{\mathrm{d}}{\longrightarrow}}
\newcommand{\convD}{\stackrel{\mathrm{D}}{\longrightarrow}}

\newcommand{\dotleq}{\stackrel{.}{\leq}}
\newcommand{\dotlt}{\stackrel{.}{<}}
\newcommand{\dotgeq}{\stackrel{.}{\geq}}
\newcommand{\dotgt}{\stackrel{.}{>}}
\newcommand{\dotdoteq}{\stackrel{\,..}{=}}

\newcommand{\eqa}[1]{\stackrel{#1}{=}}
\newcommand{\ed}{\eqa{\mathrm{d}}}
\newcommand{\lea}[1]{\stackrel{#1}{\le}}
\newcommand{\gea}[1]{\stackrel{#1}{\ge}}

\newcommand{\T}{^{\intercal}}% transpose notation
\newcommand{\setcomp}{^{\mathsf{c}}} %set complement
\newcommand{\ud}{\,\mathrm{d}} % for integrals like \int f(x) \ud x
\newcommand{\Id}{\mathrm{Id}} % identity function
\newcommand{\Bigmid}{{\ \Big| \ }}
\newcommand{\bzero}{\bm{0}}
\newcommand{\bone}{\bm{1}}

% Math functions
\newcommand{\indicator}[1]{{\bf 1}_{\braces*{#1}}}
\newcommand{\indicatore}[1]{{\bf 1}_{#1}}
\newcommand{\indicate}[1]{{\bf 1}\braces*{#1}}
\newcommand{\ofrac}[1]{{\frac{1}{#1}}}
\newcommand{\odfrac}[1]{{\dfrac{1}{#1}}}
\newcommand{\ddfrac}[2]{{\dfrac{\mathrm{d} {#1}}{\mathrm{d} {#2}}}}
\newcommand{\ppfrac}[2]{\dfrac{\partial {#1}}{\partial {#2}}}
\newcommand{\tc}[1]{^{(#1)}}

\newcommand{\bmat}[1]{\begin{bmatrix} #1 \end{bmatrix}}
\newcommand{\smat}[1]{\left[\begin{smallmatrix} #1 \end{smallmatrix}\right]}

\newcommand{\Lh}[1]{\ell_{#1}}
\newcommand{\LLh}[1]{\log{\Lh{#1}}}

% just to make sure it exists
\providecommand\given{}
% can be useful to refer to this outside \set
\newcommand\SetSymbol[2][]{%
	\nonscript\, #1#2
	\allowbreak
	\nonscript\,
	\mathopen{}}

\DeclarePairedDelimiterX\Set[2]\{\}{%
\renewcommand\given{\SetSymbol[\delimsize]{#1}}
#2
}
\DeclarePairedDelimiterX\Setc[1]\{\}{%
\renewcommand\given{\SetSymbol{:}}
#1
}

% \set{x \given f(x)=1} gives \{x : f(x)=1\}
% \set[\vert]{x \given f(x)=1} gives \{x \vert f(x)=1\}
% Starred version uses \left and \right
\NewDocumentCommand\set{s o m}{%
	\IfBooleanTF#1%
	{\IfValueTF{#2}{\Set*{#2}{#3}}{\Setc*{#3}}}%
	{\IfValueTF{#2}{\Set{#2}{#3}}{\Setc{#3}}}%
}

%\NewDocumentCommand\set{s m t| m}{%
%\IfBooleanTF#1%
%{\left\{\, #2\mathrel{} \IfBooleanTF{#3}{\middle|}{:}\mathrel{}  #4\, \right\}}%
%{\{\, #2 \IfBooleanTF{#3}{\mid}{\mathrel{} : \mathrel{}} #4\, \}}% 
%}

\NewDocumentCommand{\evalat}{s O{\big} m m}{%
\IfBooleanTF{#1}%
{{\left. #3 \right|_{#4}}}
{{#3#2|_{#4}}}%
}


\NewDocumentCommand \ifcondp {m m} {%
	#1%
	\IfValueT{#2}{\given #2}%
}

\providecommand\given{}
\DeclarePairedDelimiterXPP\cprob[1]{}(){}{
\renewcommand\given{\nonscript\,\delimsize\vert\allowbreak\nonscript\,\mathopen{}}
\ifcondp#1
}
\DeclarePairedDelimiterXPP\cexp[1]{}[]{}{
\renewcommand\given{\nonscript\,\delimsize\vert\allowbreak\nonscript\,\mathopen{}}
\ifcondp#1
}


% Allows the use of 
% \P : \mathbb{P}
% \P(X) : \mathbb{P}\left({X}\right)
% \P_{p}(X) : \mathbb{P}_{p}\left({X}\right)
% \P(X \given Y) or \P(X @| Y) or \P(X){Y} : \mathbb{P}\left({X}\, \middle| \, {Y}\right). 
% \P_{p}(X \given Y) or \P_{p}(X @| Y) : \mathbb{P}_{p}\left({X}\, \middle| \, {Y}\right)
% Caveats: Iterated expressions do not work well with \P(X @| Y) notation
% \P(\P(X @| Y) @| Z) does not work, use \P(\P(X \given Y) \given Z) 
% Starred version \P* does not use \left and \right. Maybe used in inline equations. 
\DeclareDocumentCommand \P { s e{_} >{\SplitArgument{ 1 }{ @| }}d() g } {%
	\mathbb{P}%
	\IfBooleanTF{#1}%
		{
			\IfValueT{#2}{_{#2}}%
			\IfValueTF{#4}%
				{\cprob{#3 \given #4}}%
				{\IfValueT{#3}{\cprob{#3}}}%
		}%
		{
			\IfValueT{#2}{_{#2}}%
			\IfValueTF{#4}%
				{\cprob*{#3 \given #4}}%
				{\IfValueT{#3}{\cprob*{#3}}}%
		}%
}

% Allows the use of 
% \E : \mathbb{E}
% \E[X] : \mathbb{E}\left[{X}\right]
% \E_{p}[X] or \E{p}[X] : \mathbb{E}_{p}\left[{X}\right]
% \E[X \given Y] or \E[X @| Y] or \E[X]{Y} : \mathbb{E}\left[{X}\, \middle| \, {Y}\right]. 
% \E_{p}[X \given Y] or \E_{p}[X @| Y] : \mathbb{E}_{p}\left[{X}\, \middle| \, {Y}\right]
% Caveats: Iterated expressions do not work well with \E[X @| Y] notation
% \E[\E[X @| Y] @| Z] does not work, use \E[\E[X \given Y] \given Z] 
% Starred version \E* does not use \left and \right. Maybe used in inline equations. 
\DeclareDocumentCommand \E { s e{_} >{\SplitArgument{ 1 }{ @| }}d[] g } {%
	\mathbb{E}%
	\IfBooleanTF{#1}%
		{
			\IfValueT{#2}{_{#2}}%
			\IfValueTF{#4}%
				{\cexp{#3 \given #4}}%
				{\IfValueT{#3}{\cexp{#3}}}%
		}%
		{
			\IfValueT{#2}{_{#2}}%
			\IfValueTF{#4}%
				{\cexp*{#3 \given #4}}%
				{\IfValueT{#3}{\cexp*{#3}}}%
		}%
}

% General distribution 
% E.g., \dist{Beta}[a,b][x] gives Beta(x | a,b); \dist{Beta}[a,b] gives Beta(a,b)
\ExplSyntaxOn
\NewDocumentCommand \dist {m o o} {%
\mathrm{#1}\left(%
	\IfValueT{#3}{%
		\tl_if_blank:nTF{ #3 }{\cdot\, \middle|\, }{#3\, \middle|\, }%
	}
	\IfValueT{#2}{#2}%
\right)%
}
\ExplSyntaxOff

\newcommand{\Bern}[1]{\dist{Bern}[#1]}
\newcommand{\Unif}[1]{\dist{Unif}[#1]}
\newcommand{\Dir}[1]{\dist{Dir}[#1]}
\newcommand{\Cat}[1]{\dist{Cat}[#1]}
\newcommand{\N}[2]{\dist{\calN}[#1,\, #2]}
\newcommand{\Beta}[2]{\dist{Beta}[#1,\, #2]}

\def\indep#1#2{\mathrel{\rlap{$#1#2$}\mkern5mu{#1#2}}}
\newcommand{\independent}{\protect\mathpalette{\protect\indep}{\perp}}


%Misc

% Colored underbrace/overbrace
\NewDocumentCommand {\cbrace} { D[]{black} d[] D(){\widthof{#5}} m m } {%
	\begingroup%
		\color{#1}
		\IfValueTF{#2}{%
			\overbrace{{\color{#1}#4}}^%
		}{
			\underbrace{#4}_%
		}%
		{\parbox[c]{#3}{\centering\footnotesize{#5}}}%
	\endgroup% 
}


\let\oldforall\forall
\renewcommand{\forall}{\oldforall \, }

\let\oldexist\exists
\renewcommand{\exists}{\oldexist \, }

\newcommand\existu{\oldexist! \, }


% Figures
\renewcommand{\figurename}{Fig.}
\newcommand{\figref}[1]{\figurename~\ref{#1}}
\graphicspath{{./Figures/}{./figures/}}
\pdfsuppresswarningpagegroup=1

\newcommand{\includeCroppedPdf}[2][]{%
	\IfFileExists{./Figures/#2-crop.pdf}{}{%
		\immediate\write18{pdfcrop ./Figures/#2 ./Figures/#2-crop.pdf}}%
	\includegraphics[#1]{./Figures/#2-crop.pdf}}


%%%%%%%%%%%%%%%%%%%%%%%%%%%%%%%%%%%%%%%%%%%%%%%%%%%%%%%%%%%%%%%%%%%%%%%%%

% Supplement
\newcommand{\beginsupplement}{
	\setcounter{section}{0}
	\renewcommand{\thesection}{S\arabic{section}}
	\setcounter{equation}{0}
	\renewcommand{\theequation}{S\arabic{equation}}
	\setcounter{table}{0}
	\renewcommand{\thetable}{S\arabic{table}}
	\setcounter{figure}{0}
	\renewcommand{\thefigure}{S\arabic{figure}}
}


% Editing
\definecolor{gray90}{gray}{0.9}

\ifx\nohighlights\undefined
	\newcommand{\red}[1]{{\color{red} #1}}
	\newcommand{\blue}[1]{{{\color{blue} #1}}}
	\newcommand{\msout}[1]{\text{\color{green} \sout{\ensuremath{#1}}}}
	\newcommand{\del}[1]{{\color{green}\ifmmode \msout{#1}\else\sout{#1}\fi}}
\else
	\newcommand{\red}[1]{#1}
	\newcommand{\blue}[1]{#1}
	\newcommand{\msout}[1]{#1}
	\newcommand{\del}[1]{#1}
\fi

\newcommand{\old}[1]{{\color{green} [\textrm{DELETED: }#1]}}
\newcommand{\hhide}[1]{}
%\newcommand{\hhide}[1]{{\color{magenta} [TO BE EXCLUDED] #1}}

\newcommand{\txp}[2]{\texorpdfstring{#1}{#2}}

%%%%%%%%%%%%%%%%%%%%%%%%%%%%%%%%%%%%%%%%%%%%%%%%%
% For diagnosis: if activated, will show what is causing 
% LaTeX Warning: Label(s) may have changed. Rerun to get cross-references right.

\ifx\diagnoselabel\undefined
	\relax
\else
	\makeatletter
	\def\@testdef #1#2#3{%
		\def\reserved@a{#3}\expandafter \ifx \csname #1@#2\endcsname
			\reserved@a  \else
			\typeout{^^Jlabel #2 changed:^^J%
				\meaning\reserved@a^^J%
				\expandafter\meaning\csname #1@#2\endcsname^^J}%
			\@tempswatrue \fi}
	\makeatother
\fi

%%%%%%%%%%%%%%%%%%%%%%%%%%%%%%%%%%%%%%%%%%%%%%%%%%

\def\UrlFont{\tt}


\newcounter{week}

%Theorem declarations
\newtheorem{Theorem}{Theorem}[week]
\newtheorem{Corollary}{Corollary}[week]
\newtheorem{Proposition}{Proposition}[week]
\newtheorem{Lemma}{Lemma}[week]
\newtheorem{Definition}{Definition}[week]
\newtheorem{Assumption}{Assumption}[week]

%\theoremstyle{definition}
\newtheorem{Example}{Example}[week]
\newtheorem{Remark}{Remark}[week]
\newtheorem{Exercise}{Exercise}[week]


\newcommand{\handout}[2]{
	\setcounter{week}{#1}
  \noindent
  \begin{center}
  \framebox{
    \vbox{
      \hbox to 6in {\bf An Analytical Introduction to Probability Theory \hfill}
      \vspace{5mm}
      \hbox to 6in { {\Large \hfill #1.~#2  \hfill} }
      \vspace{5mm}
      \hbox to 6in { {\em SIGNAL, NTU \hfill \small{\url{https://personal.ntu.edu.sg/wptay/}}}}
    }
  }
  \end{center}
  \setcounter{section}{0}
	\renewcommand{\thesection}{{#1}.\arabic{section}}
  \setcounter{Theorem}{0}
  \setcounter{Corollary}{0}
  \setcounter{Proposition}{0}
  \setcounter{Lemma}{0}
  \setcounter{Definition}{0}
  \setcounter{Assumption}{0}
  \setcounter{Example}{0}
  \setcounter{Remark}{0}
  \setcounter{Exercise}{0}
  \vspace*{4mm}
}

\newcommand{\calBR}{\calB(\Real)}
\newcommand{\io}{\ \mathrm{i.o.}}
\newcommand{\fo}{\ \mathrm{f.o.}}
\newcommand{\Po}[1]{\mathrm{Po}\left(#1\right)}
\externaldocument{1_BasicRealAnalysis_I}


\begin{document}

\handout{2}{Basics of Real Analysis - II}

Let $(\calX, d)$ be a metric space. 

\section{Open and Closed Sets}
\begin{Definition}
The open ball of radius $\epsilon > 0$ is defined by
\begin{align*}
B(x,\epsilon)=\set*{y \in \calX \given d(x,y)<\epsilon}.
\end{align*}
\end{Definition}

\begin{Definition}
A set $U$ is open if $\forall x \subset U$, $\exists \epsilon >0$ such that $B(x,\epsilon)\subset U$. A set $F$ is closed if $F^c = \calX\backslash F$ is open.
\end{Definition}

A set can be both open and closed, e.g., $\calX, \emptyset$. 

\begin{Example}\label{ex:discrete_X}
Let $\calX=\set{x_1, x_2, \ldots}$ be a discrete space. Consider the discrete metric
\begin{align*}
d(x,y)=\left\{
\begin{array}{ll}
0,&\ \text{if $x=y$}, \\
1,&\ \text{if $x\neq y$}.  
\end{array}\right.
\end{align*}
For any $A \subset \calX$ and $\epsilon\in(0,1)$, $B(a,\epsilon)=\{a\} \subset A$. Therefore $A$ is open. This also proves that $A$ is closed as $A^c$ is open. 
\end{Example}

If $U_i$, $i\geq1$ are open, then $\bigcup_{i=1}^\infty U_i$ is open, $\bigcap_{i=1}^n U_i$ is open but $\bigcap_{i=1}^\infty U_i$ may not be open.

\begin{Example}
Suppose that $U_i=(\frac{1}{2}-\frac{1}{i},\frac{1}{2}+\frac{1}{i})$, then $U_i$ is open while $\bigcap_{i=1}^\infty U_i=\{1/2\}$ is closed.
\end{Example}

If $F_i$ is closed, then $\bigcup_{i=1}^\infty F_i$ is closed and $\bigcap_{i=1}^\infty F_i$ is closed.

\begin{Definition}
$x$ is a limit point of $A$ if $\forall \epsilon >0$, $\exists y \in B(x,\epsilon) \cap A$, and $y\neq x$.
\end{Definition}
Therefore, $x$ is a limit point of $A$ if $\exists y_1,y_2,\ldots \in A$, $y_i\ne x$ for all $i\geq1$, s.t. $y_i \to x$.

\begin{Lemma}\label{lem:closed}
$A$ is closed if and only if all limit points of $A$ are in $A$.
\end{Lemma}
\begin{proof}
Suppose that $A$ is closed, so that $A^c$ is open. Suppose there exists a limit point $x$ of $A$ s.t.\ $x \in A^c$. Then $\exists \epsilon >0$, s.t.\ $B(x,\epsilon) \subset A^c$, which is a contradiction to $x$ being a limit point of $A$.

Suppose that all limit points of $A$ belongs to $A$. Consider $x\in A^c$. There exists $\epsilon>0$ s.t.\ $B(x,\epsilon)\subset A^c$ because otherwise there exists $(y_i)\subset A$ s.t.\ $y_i\to x$, which means that $x$ is a limit point of $A$ leading to a contradiction. This shows that $A^c$ is open and hence $A$ is closed.
\end{proof}

\section{Compact Spaces}

\begin{Definition}
$\calX$ is sequentially compact if every sequence in $\calX$ has a convergent subsequence in $\calX$.
\end{Definition}

\begin{Definition}
$\calX$ is totally bounded if $\forall \epsilon>0$, there exists a finite collection $\{ B(x_i,\epsilon) : i=1,2,\ldots,N_{\epsilon}\}$ such that  
\begin{align*}
\calX \subset \bigcup_{i=1}^{N_{\epsilon}}B(x_i,\epsilon).
\end{align*}
\end{Definition}

Note that if a set $\calX$ is totally bounded, then it is bounded. The converse is not true: consider the discrete space in \cref{ex:discrete_X}, it is bounded but not totally bounded if it is infinite.

\begin{Theorem}\label{wk1:thm:compact}
For a metric space $(\calX, d)$, the following are equivalent:
\begin{enumerate}[(i)]
\item\label{it:seq_compact} $\calX$ is sequentially compact.
\item\label{it:complete_totbdd} $\calX$ is complete and totally bounded.
\item\label{it:compact} Every open cover of $\calX$ has a finite subcover. We say that $\calX$ is compact.
\end{enumerate}
\end{Theorem}
\begin{proof}\
\begin{enumerate}[1)]
\item \ref{it:seq_compact} $\Leftrightarrow$ \ref{it:complete_totbdd}:\\
We first show that \ref{it:seq_compact} $\Rightarrow$ \ref{it:complete_totbdd}. Since $\calX$ is sequentially compact, every Cauchy sequence in $\calX$ has a convergent subsequence. From \cref{lem:Cauchy_sub}, the Cauchy sequence also converges in $\calX$, so $\calX$ is complete. Suppose that $\calX$ is not totally bounded. Then $\exists \epsilon >0$ so that $\calX$ cannot be covered by a finite collection of open balls. Choose any $x_1\in\calX$. Then $\exists x_2\notin B(x_1,\epsilon)$. Similarly, $\exists x_3\notin B(x_1,\epsilon)\cup B(x_2,\epsilon)$, and so on. The sequence $(x_n)$ does not contain any convergent subsequence since $d(x_i,x_j)\geq\epsilon$ for any $i\ne j$. This is a contraction to \ref{it:seq_compact}.

We next show that \ref{it:complete_totbdd} $\Rightarrow$ \ref{it:seq_compact}. Since $\calX$ is totally bounded, for each $m\geq 1$, there exists a finite cover $\{B(x_{m,k}, 1/m): k=1,\ldots,M_m\}$ of $\calX$. Consider any infinite sequence $(y_n)$ in $\calX$. We assume that $y_n$ are distinct because if there are infinitely many $y_n$ that are the same, then there is a trivial convergent subsequence. Then there is a $B(x_{1,k_1},1)$ that contains a subsequence $(y_{1,n})$ of $(y_n)$. Similarly, there is a $B(x_{2,k_2},1/2)$ that contains a further subsequence $(y_{2,n})$ of $(y_{1,n})$, and so on. Consider the ``diagonal'' subsequence $(y_{m,m})_{m=1}^\infty$. This sequence is Cauchy and since $\calX$ is complete, it converges.


\item \ref{it:seq_compact} $\Leftrightarrow$ \ref{it:compact}:\\
We show \ref{it:compact} $\Rightarrow$ \ref{it:seq_compact}. To do that, we first prove the following facts:
\begin{enumerate}
	\item Any compact $A \subset \calX$ is closed. In particular, if $\calX$ is compact, then it is closed.\\
	Let $x\in A^c$ and $U_n = \{y : d(y,x) > 1/n\}$ for $n\geq 1$. Every $y\in\calX$ with $y\ne x$ has $d(y,x)>0$ so $y$ belongs to some $U_n$. Therefore, $\{U_n : n\geq 1\}$ covers $A$ and there must be a finite subcover. Let $N$ to be the largest index in the subcover, i.e., every $y\in A$ lies in some $U_n$ where $n\leq N$. Then $B(x,1/N) \subset A^c$ and $A$ is closed.
	\item If $\calX$ is compact and $A\subset\calX$ is closed, then $A$ is compact.\\
	Let $\{U_n\}$ be an open cover of $A$. Then $\{U_n\}\cup\{ A^c\}$ is an open cover of $\calX$. There is a finite subcover, say, $\{U_1,\ldots,U_N,A^c\}$ of $\calX$. Then $\{U_1,\ldots,U_N\}$ is a finite open cover of $A$.
\end{enumerate}
Suppose $\calX$ is compact. Assume there is a sequence $(x_n)$ that has no convergent subsequences. In particular, this sequence has infinitely distinct points $y_1, y_2, \ldots$. Since there is no convergent subsequence, there is some open ball $B_k$ containing each $y_k$ and no other $y_i$. The set $A=\{y_1,y_2,\ldots\}$ is closed as it has no limit points, so it is compact. But $\{B_k\}$ is an open cover of $A$ and has no finite subcover, a contradiction. Therefore $(x_n)$ has a convergent subsequence whose limit lies in $\calX$ as $\calX$ is closed.

We now show \ref{it:seq_compact} $\Rightarrow$ \ref{it:compact}. Suppose that $\calX$ is sequentially compact. Let $\{G_\alpha\subset \calX : \alpha \in I\}$ be an open cover of $\calX$. We claim that there exists $\epsilon > 0 $ such that every ball $B(x,\epsilon)$ is contained in some $G_\alpha$. Suppose not. Then for each positive integer $n$, $\exists y_n\in\calX$ such that $B(y_n,1/n)$ is not contained in any $G_\alpha$. By hypothesis, there exists a subsequence $y_{n_i} \to y \in\calX$. Since $\{G_\alpha\subset \calX : \alpha \in I\}$ is an open cover of $\calX$, there exists $G_{\alpha_0} \ni y$ and $\epsilon>0$ such that $B(y,\epsilon) \subset G_{\alpha_0}$. Choose $n_i$ sufficiently large so that $d(y_{n_i},y) < \epsilon/2$ and $1/n_i<\epsilon/2$. Then $B(y_{n_i},1/n_i)\subset G_{\alpha_0}$, a contradiction.

Since $\calX$ is sequentially compact, it is totally bounded, so there exists a finite collection of balls of radius $\epsilon$ s.t. $\{B(x_i,\epsilon):i=1,2,\ldots,N\}$ covers $\calX$. Choose $\alpha_i \in I$ s.t. $B(x_i,\epsilon)\subset G_{\alpha_i}$. Then $\{G_{\alpha_i}:i=1,2,\ldots,N\}$ is a finite subcover of $\calX$, which means $\calX$ is compact.
\end{enumerate}



\end{proof}

\begin{Theorem}[Heine-Borel]\label{Theorem:Heine-Borel} 
$A \subset \Real^k$ is compact iff $A$ is closed and bounded.
\end{Theorem}

\begin{proof} 
Since the following proof can be repeated in every dimension, it suffices to prove only for $A \subset \Real$.

We first prove sufficiency. Suppose $A$ is closed and bounded and $(x_n)\subset A$. Then $x_n$ is bounded since $A$ is bounded. By \cref{lem:bdd_conv}, there exists convergent $(x_{n_i})$ such that $x_{n_i}\to x\in\Real$. Since $A$ is closed, $x \in A$. Therefore, $A$ is compact.

Next, suppose that $A$ is compact. Then from \cref{wk1:thm:compact}, $A$ is complete and totally bounded, hence closed and bounded in $\Real$.

%We next show necessity. Suppose that $A$ is compact, and hence sequentially compact. For $x_{n}\to x$, $\exists (x_{n_i})$, such that $x_{n_i}\to x'\in A$. By \cref{wk1:subseq}, $x_{n}\to x'$ and $x'=x$ from \cref{wk1:unique}. So $A$ is closed. If $A$ is not bounded, then $\exists a_0 \in A$ and $(a_n) \subset A$ with $d(a_n,a_0) > d(a_{n-1},a_0)+1$ for $n\geq1$. This sequence cannot have a convergent subsequence since for $m>n$, we have $d(a_n,a_m) \geq d(a_m,a_0)-d(a_n,a_0) > 1$, which is a contradiction to the fact that $A$ is compact. Therefore, $A$ is bounded.
\end{proof}


\section{Continuity}

\begin{Definition}
A function $f: (\calX, d_{\calX}) \mapsto (\calY, d_{\calY})$ is (pointwise) continuous at $x \in \calX$ if $\forall \epsilon>0$, $\exists \delta_x>0$ s.t.\ $d_{\calY}(f(x),f(z))< \epsilon$, whenever $d_{\calX}(x,z)<\delta_x$. 
\end{Definition}

Note that the definition is equivalent to saying that $\forall \epsilon>0$, $\exists \delta_x>0$ s.t. $B(x,\delta_x) \subset f^{-1}(B(f(x),\epsilon))$. 

\begin{Lemma}
$f: (\calX, d_X) \mapsto (\calY, d_Y)$ is continuous iff $f^{-1}(U)$ is open in $\calX$ for every open $U \subset \calY$.
\end{Lemma}
\begin{proof}
Suppose $f$ is continuous and $U$ is open in $\calY$. For each $y\in U$, there exists an open ball $B(y,r) \subset U$. Since $f$ is continuous, for each $x \in f^{-1}(\set{y})$, $\exists \delta_x>0$ s.t. $B(x,\delta_x) \subset f^{-1}(B(y,r)) \subset f^{-1}(U)$. Therefore, $f^{-1}(U) = \bigcup_{x\in f^{-1}(U)} B(x,\delta_x)$ is open. 

To prove the other direction, we have for every $x\in\calX$ and $\epsilon>0$, $f^{-1}(B(f(x),\epsilon))$ is open and contains $x$, so it contains an open ball around $x$. Thus, $f$ is continuous at $x$. 
\end{proof}

\begin{Lemma}
If $f$ is continuous and $B\in \calX$ is compact, then $f(B)\triangleq\{f(x): x \in B\}$ is compact.
\end{Lemma}
\begin{proof}
Consider $y_n= f(x_n)$. $\exists (x_{n_i})$ s.t. $x_{ni}\to x \in B$. From the continuity of $f$, $y_{n_i}=f(x_{n_i}) \to f(x) \in f(B)$. 
\end{proof}

From \cref{Theorem:Heine-Borel}, $f(B)$ is closed and bounded. Therefore, $\sup_{x\in B} f(x)$ and $\inf_{x\in B} f(x)$ are both achieved on $B$.

\begin{Definition}
$f$ is uniformly continuous if $\forall \epsilon>0$, $\exists \delta>0$ s.t. $d_{\calY}(f(x),f(z))<\epsilon$, $\forall d_{\calX}(x,z)<\delta$.
\end{Definition}

\begin{Example}\
\begin{enumerate}[(a)]
	\item If $|f(x)-f(y)| < L d(x,y)$ for some positive constant $L$ and all $x,y \in \calX$, we say that $f$ is Lipschitz continuous. It is clear that $f$ is uniformly continuous on $\calX$. 
	\item If $f : \Real\mapsto\Real$ is differentiable with $\sup |f'| <\infty$, where $f'$ is the derivative of $f$, then $f$ is uniformly continuous from the mean value theorem.
\end{enumerate}
\end{Example}


\begin{Lemma}
Suppose that $f$: $K \mapsto \calY$, where $K \subset \calX$ is compact. If $f$ is continuous, then $f$ is uniformly continuous.
\end{Lemma}
\begin{proof}
Fix $\epsilon>0$. For each $x \in \calX$, $\exists \delta_x>0$, s.t. $d_{\calY}(f(x),f(y))< \epsilon/2$ whenever $d_{\calX}(x,y)<\delta_x$. We have $\{B(x,\delta_x/2) : x\in K\}$ is an open cover of $K$. Since $K$ is compact, exists a finite subcover $\{B(x_i,\delta_{x_i}/2): 1\leq i \leq n\}$. Let $\delta= \min\{\delta_{x_i},\cdots, \delta_{x_n}\}>0$. 

For each $x \in K$, $\exists x_i$, s.t.\ $d_{\calX}(x,x_i)<\delta_{x_i}/2$. Then for all $y$ s.t.\ $d_{\calX}(x,y)<\delta/2$, we have 
\begin{align*}
d_{\calX}(y,x_i) &\leq d_{\calX}(x,y)+d_{\calX}(x,x_i)<\delta_{x_i}.
\end{align*}
Therefore, we obtain
\begin{align*}
d_{\calY}(f(x),f(y))&\leq d_{\calY}(f(x),f(x_i))+d_{\calY}(f(x_i),f(y)) \leq \epsilon,
\end{align*}
which shows that $f$ is uniformly continuous.
\end{proof}

Note that in the above proof, $y$ need not be in $K$. We obtain a slightly stronger result here.

\begin{Corollary}\label{wk2:f_K_compact}
Suppose $f:\calX\mapsto\calY$ is continuous, and $K\subset\calX$ is compact. Then $\forall \epsilon>0$, $\exists \delta>0$ such that $d_{\calY}(f(x),f(y))< \epsilon$ whenever $x \in K$, $y \in \calX$ and  $d_{\calX}(x,y)< \delta$.
\end{Corollary}

\section{Riemann Integral}

We consider a function $f:[a,b]\mapsto\Real$ in this section. A partition $P=(x_0,\ldots,x_N)$ is defined by $x_0=a<x_1<x_2<\cdots<x_N=b$. We say that $Q$ is a refinement of $P$ if $P\subset Q$. Let $\calP$ be the collection of all partitions. For $P\in\calP$, define
\begin{align*}
U(f,P)&=\sum_{i=1}^N\sup_{[x_{i-1},x_i]}\mathllap{f}\cdot(x_i-x_{i-1}), \\
L(f,P)&=\sum_{i=1}^N\inf_{[x_{i-1},x_i]}\mathllap{f}\cdot(x_i-x_{i-1}). 
\end{align*}

\begin{Lemma}
$L(f,P)\leq U(f,Q)$, $\forall P,Q \in \calP$
\end{Lemma}
\begin{proof}
\begin{align*}
L(f,P)&\leq L(f,P\cup Q)\quad \text{since $\inf_{I_1} f \leq \inf_{I_2} f$ if $I_1 \supset I_2$}\\
&\leq U(f,P\cup Q)\\
&\leq U(f,Q)\quad \text{since $\sup_{I_1} f \geq \sup_{I_2} f$ if $I_1 \supset I_2$}.
\end{align*}
\end{proof}

From the above lemma, we have
\begin{align}\label{L_U}
\sup_{P\in\calP}{L(f,P)}\leq\inf_{P\in\calP}{U(f,P)}
\end{align}

\begin{Definition}
$f:[a,b]\mapsto \Real$ is Riemann integrable if equality in \cref{L_U} holds, i.e.,
\begin{align*}
\forall \epsilon >0,\ \exists P \in \calP\ \st\ U(f,P)-L(f,P)<\epsilon.
\end{align*}
\end{Definition}

\begin{Example}
The following function is not Riemann integrable as $L(f,P)=0$ and $U(f,P)=1$ for all $P\in\calP$:
\begin{align*}
f(x)=\left\{
 \begin{array}{ll}
 1,&\ \text{if $x\in\bbQ\cap[0,1]$}, \\
 0,&\ \text{otherwise}.  
 \end{array}
\right.
\end{align*}
\end{Example}

\begin{Definition}
A set $A \subset \Real$ has Lebesgue measure zero if $\forall\epsilon> 0$, there exists open intervals $(\alpha_1,\beta_1),(\alpha_2,\beta_2),\ldots$ s.t. 
\begin{align*}
A \subset \bigcup_{i=1}^{\infty}(\alpha_i,\beta_i)\ \text{and}\ \sum_{i=1}^{\infty}(\alpha_i-\beta_i)<\epsilon.
\end{align*}
\end{Definition}

If a countable sequence of sets $A_1,A_2,\ldots$ each of which has Lebesgue measure zero, then the union $\bigcup^{\infty}_{i=1}{A_i}$ has Lebesgue measure zero. To see this, let $\epsilon>0$ and $A_j$ be covered by $\bigcup_i{(\alpha_{ij},\beta_{ij})}$ with $\sum_i(\alpha_{ij}-\beta_{ij})<\epsilon/2^j$.

\begin{Theorem}[Henri Lebesgue]\label{thm:RiemannIntegrable}
Suppose $f:[a,b]\mapsto \Real$ is bounded. Then $f$ is Riemann integrable iff $\exists A \subset [a,b]$ of Lebesgue measure zero s.t. $f$ is continuous on $[a,b]\backslash A$.\footnote{$f$ is also said to be continuous almost everywhere on $[a,b]$.}
\end{Theorem}
\begin{proof}
We first show that if $f$ is Riemann integrable, then its set of discontinuities has Lebesgue measure zero. Observe that $y \in (a,b)$ is a point of discontinuity of $f$ iff $\exists j \in \mathbb{Z}_{+}$ s.t.\ $\sup_{I}{f}-\inf_{I}{f}\geq 1/j$ for all open intervals $I \subset (a,b)$ containing $y$. Let 
\begin{align*}
S_j=\set*{y\in(a,b) \given \sup_I{f}-\inf_I{f}\geq \frac{1}{j}\ \forall\ \text{open intervals $I\subset(a,b)$ with $y\in I$}}.
\end{align*}
Then, the set of discontinuities of $f$ in $(a,b)$ is $\bigcup^{\infty}_{j=1}{S_j}$. For $\epsilon >0$, since $f$ is Riemann integrable, there exists some partition $P=(x_0,\ldots,x_N)$ s.t.
\begin{align}\label{wk2:U-L}
U(f,P)-L(f,P)=\sum_{i=1}^N (\sup_{[x_{i-1},x_i]}\mathllap{f}-\inf_{[x_{i-1},x_i]}\mathllap{f})\cdot(x_i-x_{i-1}) < \frac{\epsilon}{j}.
\end{align}
Let $B=\{i:(x_{i-1},x_i)\cap S_j\neq\emptyset\}$. Then from \cref{wk2:U-L}, we have
\begin{align}\label{wk2:U-L1}
\sum_{B}{(\sup_{[x_{i-1},x_i]}\mathllap{f}-\inf_{[x_{i-1},x_i]}\mathllap{f})\cdot(x_i-x_{i-1})}+\sum_{B^c}{(\sup_{[x_{i-1},x_i]}\mathllap{f}-\inf_{[x_{i-1},x_i]}\mathllap{f})\cdot(x_i-x_{i-1})}<\frac{\epsilon}{j}
\end{align}
Since 
\begin{align*}
\sum_{B^c}{(\sup_{[x_{i-1},x_i]}\mathllap{f}-\inf_{[x_{i-1},x_i]}\mathllap{f})\cdot(x_i-x_{i-1})}\geq 0
\end{align*}
and
\begin{align*}
\sum_{B}{(\sup_{[x_{i-1},x_i]}\mathllap{f}-\inf_{[x_{i-1},x_i]}\mathllap{f})\cdot(x_i-x_{i-1})}\geq \frac{1}{j}\sum_{i\in B}{(x_i-x_{i-1})},
\end{align*}
we obtain from \cref{wk2:U-L1},
\begin{align*}
\sum_{i\in B}{(x_i-x_{i-1})}<\epsilon.
\end{align*}
We have
\begin{align*}
S_j\subset \bigcup_{i\in B}(x_{i-1},x_i)\bigcup\{x_0,x_1,\cdots,x_N\},
\end{align*}
therefore $S_j$ has Lebesgue measure zero.

We next prove the converse. Fix an $\epsilon>0$. Assume that we have $A$ with Lebesgue measure zero, which means that there is a cover $\bigcup_{j=1}^{\infty}{(\alpha_j,\beta_j)} \supset A$ s.t.\ $\sum_{j=1}^{\infty}(\beta_j-\alpha_j)<\epsilon$.  Let $K=[a,b]\backslash\bigcup_{j\geq 1}{(\alpha_j,\beta_j)}$, which is closed and bounded and therefore compact by \cref{Theorem:Heine-Borel}. Since $f$ is continuous, from \cref{wk2:f_K_compact}, $\exists \delta>0$, s.t.\ $|f(x)-f(y)|<\epsilon$ whenever $x \in K$, $y\in[a,b]$ and $|x-y|<\delta$. 

We choose a partition $P$ with $a=x_0<x_1<x_2<\cdots<x_N=b$ s.t.\ $\max_{1\leq i \leq N}(x_i-x_{i-1})<\delta$. If $[x_{i-1},x_i]\cap K=\emptyset$, then $[x_{i-1},x_i]\subset \bigcup_{j}{(\alpha_j,\beta_j)}$ and
\begin{align*}
\sum_{i:[x_{i-1},x_i]\cap K=\emptyset}{\parens{\sup_{[x_{i-1},x_i]}\mathllap{f}-\inf_{[x_{i-1},x_i]}\mathllap{f}}\cdot(x_i-x_{i-1})}&\leq (\sup_{[a,b]}{f}-\inf_{[a,b]}{f})\cdot \sum_{j}{(\beta_j-\alpha_j)} < M\epsilon,
\end{align*}
where $M=\sup_{[a,b]}{f}-\inf_{[a,b]}{f} < \infty$. Suppose $[x_{i-1},x_i]\cap K\ne\emptyset$. Then for any $y, z \in [x_{i-1},x_i]$ and $y_i\in [x_{i-1},x_i]\cap K$, we have 
\begin{align*}
|f(y)-f(z)|&\leq|f(y)-f(y_i)|+|f(y_i)-f(z)|\\
&<\epsilon+\epsilon\\
&=2\epsilon,
\end{align*}
where the last inequality follows because $|y-y_i|,\ |z-y_i| < \delta$. Therefore,
\begin{align*}
\sum_{i: [x_{i-1},x_i]\cap K\ne\emptyset} (\sup_{[x_{i-1},x_i]}\mathllap{f}-\inf_{[x_{i-1},x_i]}\mathllap{f})\cdot(x_i-x_{i-1}) \leq 2\epsilon (b-a).
\end{align*}
We finally obtain
\begin{align*}
U(f,P)-L(f,P)\leq M\epsilon + 2\epsilon(b-a)= (M+2(b-a))\epsilon, 
\end{align*}
and the proof is complete.
\end{proof}

From \cref{thm:RiemannIntegrable}, we see that only a very limited class of functions $f$ is Riemann integrable. This is not sufficient to model many practical applications. Therefore, Henri Lebesgue, a French mathematician in the 17th century, embarked on a program to introduce a much more versatile integral known as the Lebesgue integral. We will introduce this in the coming week as part of the theory of probability. 

%\bibliography{mybib}
%\bibliographystyle{alpha}




\end{document}
