\documentclass[../aipt.tex]{subfiles}
\graphicspath{{\subfix{../Figures/}}}



\begin{document}

\handout{1}{Basics of Real Analysis - I}

\section{Introduction}
This basic real analysis introduction covers topics in sequences, compact spaces and continuity. Recommended textbook is ``Principles of Mathematical Analysis'' by Walter Rudin. 

\section{Sequences}
Inner product space $\subset$ normed space $\subset$ metric space $\subset$ topological space. Working with metric spaces suffices for our purposes.

\begin{Definition}
A metric space is an ordered pair $\left(\calX, d\right)$ where $\calX$ is a set and $d$ is a metric on $\calX$, such that $\forall x, y, z \in \calX$, the following holds:
\begin{enumerate}[1)]
\item $d(x,y) \geq 0$, where equality holds iff $x=y$.
\item $d(x,y)=d(y,x)$.
\item Triangle inequality: $d(x,y) \leq d(x,z) + d(y,z)$.
\end{enumerate}
\end{Definition}

\begin{Example} Examples of metric space:
\begin{itemize} 
\item $\left(\bbR, |x-y|\right)$ (normed space)
\item $\left(\bbR^k, \norm{x-y}_p\right)$, where $p \geq 1$ (normed space)
\item $\ell^p(\Real) = \set*{ x = (x_n)_{n\geq1} \given \norm{x}_p=\parens*{\sum_{n\geq1} |x_n|^p}^{1/p} < \infty}$ (normed space)
\item $\parens{\calX, d}$, where $d(x,y)=0$ if $x=y$ and $d(x,y)=1$ otherwise.
\end{itemize}
\end{Example}

We write $(x_n)$ or $(x_n)_{n\geq1}$ as shorthand for the sequence $x_1, x_2, \ldots$.

\begin{Definition}
For $(x_n) \subset \calX$, we say that $x_n \to x$ as $n\to\infty$ or $\lim_{n\to\infty} x_n=x$ if 
\begin{align*}
\forall \epsilon>0,\ \exists N\ \ST d(x_n, x) < \epsilon,\ \forall n \geq N.
\end{align*}
\end{Definition}

\begin{Definition}
We say $x_n$ diverges to $\infty$ if there exists $x_0\in\calX$ and
\begin{align*}
\forall K\geq 0,\ \exists N\ \ST d(x_n,x_0) \geq K,\ \forall n\geq N.
\end{align*}
\end{Definition}

We define extended real numbers $\overline{\Real} = \bbR \cup \{-\infty, \infty\}.$ For $x \in \bbR$, we have:
\begin{enumerate}[1)]
\item $x + \infty = \infty$
\item $x - \infty = -\infty$
\item $x \cdot \pm\infty = \pm\infty, \forall x > 0$
\item $0 \cdot \pm\infty = 0$ (by convention)
\end{enumerate}

The following lemmas are based on $(\Real, \abs{\cdot})$ but most can be generalized to $(\calX,d)$. 

\begin{Lemma}\label{wk1:unique}
If $(a_n)$ converges, then its limit is unique.
\end{Lemma}

\begin{proof}
Supposing $a_n \to a$ and $a_n \to a^\prime$, we have
\begin{align*}
\forall \epsilon > 0,\ \exists N\ \ST &|a_n - a| < \epsilon \text{ and } |a_n - a^\prime| < \epsilon,\ \forall n \geq N.
\end{align*}
By the triangle inequality, we obtain:
\begin{align*}
|a - a^\prime| \leq |a_n - a| + |a_n - a^\prime| < 2\epsilon.
\end{align*}
Since $\epsilon$ is arbitrary, $a = a^\prime$. 
\end{proof}

\begin{Lemma}\label{lem:conv_bdd}
If $(a_n)$ converges, $|a_n| < \infty$.
\end{Lemma}

\begin{proof}
Supposing $a_n \to a$, $\exists N$, $\ST |a_n - a| < 1$, $\forall n \geq N$. Then we obtain
\begin{align*}
|a_n| &\leq |a| + 1, \forall\ n \geq N, \\
\end{align*}
and
\begin{align*}
|a_n| &\leq \max \{a_1, \ldots, a_{N-1}, |a|+1\} < \infty.
\end{align*} 
\end{proof}

\begin{Lemma}\label{lem:inc_bdd}
If $(a_n)$ is increasing (i.e., $a_{n+1} \geq a_n$ for all $n\geq1$) and is bounded above, then $(a_n)$ converges.
\end{Lemma}
\begin{proof}
To prove the lemma, we need some definitions. We define $\sup{A}$ = least upper bound (LUB) of $A$ if $ A\neq \emptyset$ and $A \subset\Real$, i.e., 
\begin{enumerate}[1)]
\item $\forall a \in A$, $a \leq \sup{A}$.
\item if $\gamma < \sup{A}$, $\exists a \in A$, $\ST \gamma < a$. \label{item_supA}
\end{enumerate}
For the definition to be well-defined, we need the \rm{LUB/Completeness axiom}: $\sup{A}$ exists in $\Real$ for all $A\ne\emptyset$ bounded above. (Note that this axiom does not hold for $\bbQ$.) If $A$ is not bounded above, we set $\sup A = \infty$. From \cref{item_supA}, we obtain a useful property:
\begin{align*}
\forall \epsilon > 0,\ \exists a \in A,\ \ST \sup{A} - \epsilon < a.
\end{align*}
The greatest lower bound of $A$ is written as $\inf A$.

Returning to the proof of the lemma, let $a = \sup{a_n} < \infty$ since $(a_n)$ is bounded. We have 
\begin{align*}
&\forall \epsilon > 0,\ \exists N,\ \ST a - \epsilon \leq a_N \leq a_n \leq a,\ \forall n \geq N. \\
&\implies a_n \to a.
\end{align*}
\end{proof}
Applying \cref{lem:inc_bdd} to the negative of a sequence, we also conclude that any decreasing sequence that is bounded below converges.

\begin{Remark}\
\begin{enumerate}
	\item If $a_n\in \overline{\Real}_+$ and is increasing, then $\lim_{n\to\infty} a_n$ exists (maybe $\infty$).
	\item If $a_n\in\Real_+$, then from \cref{lem:inc_bdd}, $\sum_{k=1}^\infty a_k \triangleq \lim_{n\to\infty} \sum_{k=1}^n a_k$ exists. 
\end{enumerate}
\end{Remark}

\begin{Definition}
A subsequence of $(a_n)$ is a sequence $(a_{n_i})$, where $n_1 < n_2 < \ldots$ is an increasing sequence of indices.
\end{Definition}

\begin{Lemma}\label{wk1:subseq}
If $a_n \to a \iff a_{n_i} \to a$, for all subsequences $(a_{n_i})$.
\end{Lemma}
\begin{proof}
The direction ``$\Leftarrow$'' is obvious. To prove the other direction, we have $\forall \epsilon > 0$, $\exists N$ $\ST |a_n - a| < \epsilon,\ \forall n \geq N$. Therefore, $\forall n_i \geq N$, $|a_{n_i} - a| < \epsilon$.
\end{proof}

\begin{Lemma}\label{lem:mono_sub} 
Every sequence $(a_n)$ in $\bbR$ has a monotone subsequence.
\end{Lemma}
\begin{proof}
Suppose there are no increasing subsequences. Then, $\exists n_1$, $\ST a_{n_1} \geq a_n$, $\forall n \geq n_1$. Similarly, $\exists n_2 > n_1$, $\ST a_{n_2} \geq a_n$, $\forall n \geq n_2$, and so on. This constructs a decreasing subsequence $a_{n_1} \geq a_{n_2} \geq \ldots$. 
\end{proof}

\begin{Lemma}\label{lem:bdd_conv}
Every bounded sequence in $\bbR$ contains a convergent subsequence.
\end{Lemma}
\begin{proof}
From \cref{lem:mono_sub}, a bounded sequence has a monotone subsequence, which converges from \cref{lem:inc_bdd}.
\end{proof}

Let $S =$ set of subsequence limits of $(a_n) \subset \overline{\Real}$. We define $\limsup_{n \to \infty} {a_n}=\sup{S}$, and $\liminf_{n \to \infty}{a_n}=\inf{S}$. We have
\begin{align*}
\forall \epsilon > 0,\ \exists s \in S,\ \ST \sup{S} - \epsilon/2 \leq s \leq \sup{S}.
\end{align*}
Let $a_{n_i} \to s$ (can you see why such a subsequence must exist?) so that
\begin{align*}
\forall \epsilon > 0,\ \exists N,\ \ST |a_{n_i} - s| < \epsilon/2,\ \forall n_i > N.
\end{align*}
Therefore,
\begin{align*}
&|a_{n_i} - \sup{S}| \leq |a_{n_i} - s| + |s - \sup{S}| < \epsilon,
\end{align*}
which implies that $\sup{S} \in S$ and $\sup{S} = \max{S}$, i.e., $\limsup_{n\to\infty} a_n$ exists. Similarly, we have $\inf{S} = \min{S}$.

\begin{Lemma}
Suppose $s = \limsup_{n \to \infty}{a_n} > -\infty$. Then $\forall \epsilon > 0$, $\exists N$ $\ST a_n \leq s + \epsilon$, $\forall n \geq N$.
\end{Lemma}
\begin{proof}
If $s = \infty$, the lemma obviously holds. Suppose $s < \infty$. We prove by contradiction. Suppose $\exists \epsilon > 0$ and subsequence $(a_{n_i})$ $\ST a_{n_i} > s + \epsilon$. Since $s < \infty$, $(a_{n_i})$ is bounded. Then from \cref{lem:bdd_conv}, $\exists \text{subsubsequence } (a_{n_i'})$, $\ST a_{n_i'} \to a' \geq s + \epsilon > s$, a contradiction to the definition of $s$.
\end{proof}


\begin{Lemma}\label{wk1:limsup_infsup}
$\displaystyle\limsup_{n \to \infty} a_n = \inf_{n\geq1} \sup_{m\geq n}{a_m}$ and $\displaystyle\liminf_{n \to \infty}a_n = \sup_{n\geq1} \inf_{m\geq n}{a_m}$.
\end{Lemma}
\begin{proof}
If $\limsup_{n \to \infty} a_n = \pm \infty $, then the result clearly holds. Suppose $-\infty < \limsup_{n \to \infty} a_n = \sup S < \infty$, where $S$ is the set of subsequential limits.  

For any $\epsilon > 0$, there exists a subsequence $(a_{n_i})$ $\ST \sup{S} - \epsilon \leq a_{n_i}$. Then,
\begin{align*}
\sup{S} - \epsilon &\leq \sup_{m \geq n}{a_m},\ \forall n \geq n_1. 
\intertext{Letting $n\to\infty$, we have}
\sup{S} - \epsilon &\leq \lim_{n \to \infty}\sup_{m \geq n}{a_m} = \inf_{n\geq1} \sup_{m\geq n}{a_m},
\end{align*}
since $\sup_{m \geq n}{a_m}$ is a decreasing sequence in $n$. As $\epsilon$ is arbitrary, we have
\begin{align*}
\sup S \leq \inf_{n\geq1} \sup_{m\geq n}{a_m}.
\end{align*}
On the other hand, for each $n\geq 1$, $\exists a_{m_n} \geq \sup_{m \geq n}{a_m} - 1/n$. Then we have
\begin{align*}
\sup{S} \geq \lim_{n \to \infty} a_{m_n} \geq \inf_{n \geq 1} \sup_{m \geq n}{a_m}. 
\end{align*}
The other claim is proved similarly and the proof is now complete.
\end{proof}


\section{Cauchy Sequences}
\begin{Definition}
A sequence $(a_n)$ is a Cauchy sequence if $\forall \epsilon > 0$, $\exists N$ $\ST d(a_n, a_m) < \epsilon$, $\forall n,m \geq N$.
\end{Definition}

\begin{Lemma}
If $(a_n)$ converges, then it is Cauchy.
\end{Lemma}
\begin{proof}
Suppose $a_n\to a$. Then $\forall \epsilon >0$, $\exists N\ \ST d(a_n,a) < \epsilon/2\ \forall n\geq N$. Then $\forall n,m \geq N$, we have
\begin{align*}
d(a_n, a_m) \leq d(a_n,a) + d(a_m,a) < \epsilon.
\end{align*}
\end{proof}

\begin{Definition}
A metric space $(\calX, d)$ in which every Cauchy sequence converges in $\calX$ is complete.
\end{Definition}

Example of an incomplete space is $\Rat$, the set of rational numbers: for every irrational number, one can construct a sequence of rational numbers that converges to it.


\begin{Theorem}\label{wk1:Rcomplete}
$\Real^k$ is complete.
\end{Theorem}
\begin{proof}
WLOG, we prove for $\Real$. We first show 2 lemmas.

\begin{Lemma}\label{lem:Cauchy_bdd}
A Cauchy sequence is bounded.
\end{Lemma}
\begin{proof}
Similar to \cref{lem:conv_bdd}.
\end{proof}

\begin{Lemma}\label{lem:Cauchy_sub}
If $(a_n)$ is Cauchy and there is a subsequence $a_{n_i} \to a$, then $a_n \to a$.
\end{Lemma}
\begin{proof}
For any $\epsilon > 0$, $\exists N$ such that $\forall n, n_i \geq N$,
\begin{align*}
d(a_n, a) \leq d(a_n, a_{n_i}) + d(a_{n_i},a) \leq \epsilon.
\end{align*}
\end{proof}

We now return to the proof of \cref{wk1:Rcomplete}. Let $(a_n)$ be a Cauchy sequence in $\Real$. Then $(a_n)$ is bounded from \cref{lem:Cauchy_bdd}. From \cref{lem:bdd_conv}, there exists a convergent subsequence. \Cref{lem:Cauchy_sub} then implies $(a_n)$ converges and the proof is complete.
\end{proof}

%\bibliography{mybib}
\bibliographystyle{alpha}

\end{document}
